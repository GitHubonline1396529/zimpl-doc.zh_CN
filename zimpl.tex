% dvips -ta4 -O0in,-1in zimpl.dvi
%* * * * * * * * * * * * * * * * * * * * * * * * * * * * * * * * * * * * * * *
%*                                                                           *
%*   File....: zimpl.tex                                                     *
%*   Name....: Zuse Institute Mathematical Programming Language              *
%*   Author..: Thorsten Koch                                                 *
%*   Copyright (C) 2010-2020 by Author, All rights reserved                  *
%*                                                                           *
%* * * * * * * * * * * * * * * * * * * * * * * * * * * * * * * * * * * * * * *
%
% \documentclass[11pt]{article}
\documentclass[11pt]{ctexart} % Use CTeX to display Chinese.
% for formal Chinese articles, It's better to use the following geometry.
\usepackage[
  a4paper,
  margin=1in
]{geometry}
%\renewcommand{\rmdefault}{pmnj}
%\renewcommand{\ttdefault}{pcr}
%\renewcommand{\sfdefault}{pmy}
\usepackage[T1]{fontenc}
\usepackage{textcomp}
\usepackage[small,euler-digits]{eulervm}
% \usepackage{a4}
% \usepackage[latin1]{inputenc}
\usepackage[utf8]{inputenc}
\usepackage{amsmath}
\usepackage{amssymb}
\usepackage{xspace}
\usepackage{epsfig}
\usepackage{fancyhdr}
%\usepackage{xspace}
\usepackage{multicol}
\usepackage{url}
\usepackage{color}
\usepackage{booktabs}
\usepackage{listings}
\usepackage{graphicx}
%\usepackage{zibtitlepage}
%\usepackage{pdfdraftcopy}
\usepackage[%
%dvips,
% XeTeX Conflict with dvips so if I use CTeX `dvips` should be commented
bookmarks,
pdffitwindow,
pdfcenterwindow=true,
pdfstartview=Fit
]{hyperref}
\hypersetup{%
pdftitle={非官方中文翻译版ZIMPL用户指南},
pdfsubject={楚泽研究所数学规划设计语言版本3.7.0},
pdfauthor={托尔斯·滕科赫},
pdfkeywords={数学建模语言, 数学规划, 优化, 代数建模语言}
}
%
\definecolor{seagreen}{rgb}{0.18,0.74,0.56}
\definecolor{darkgreen}{rgb}{0.0,0.35,0.00}
\definecolor{navyblue}{rgb}{0.0,0.0,0.5}
\definecolor{steelblue}{rgb}{0.27,0.51,0.71}
\definecolor{siennabrown}{rgb}{0.63,0.32,0.18}
\definecolor{firebrickred}{rgb}{0.69,0.13,0.13}
\definecolor{gray75}{rgb}{0.75,0.75,0.75}
%
\lstloadlanguages{C}
\lstdefinelanguage{mps}{%
   keywords={NAME,ROWS,COLUMNS,RHS,BOUNDS,ENDATA},%
   sensitive,%
   keywordstyle=\color{navyblue},%
}[keywords]%
%
\lstdefinelanguage{YACC}{%
   keywords={\%token,\%type,\%left,\%right,\%union},%
   sensitive,%
   singlecomment={/*}{*/},%
   stringizer=[b]',%
   keywordstyle=\color{navyblue},%
   commentstyle=\color{darkgreen},%
   stringstyle=\color{steelblue}%
}[keywords,comments,strings]%
%
\lstdefinelanguage{zimpl}{%
   keywords={set,var,param,minimize,maximize,subto},%
   ndkeywords={read,as,comment,binary,integer,real,sum,forall,do,in,proj,vif,vabs,and,or,then,else,end},%
   sensitive,%
   showstringspaces=false,%
   morecomment=[l]\#,
   morestring=[b]",%
   keywordstyle=\color{red},%
   ndkeywordstyle=\color{navyblue},%
   commentstyle=\color{darkgreen},%
   stringstyle=\color{steelblue}%
}[keywords,comments,strings]%
%
\lstdefinestyle{myc}{%
   basicstyle=\sffamily\footnotesize,%
   numberstyle=\sffamily\tiny\color{siennabrown},stepnumber=1,%
   keywordstyle=\color{navyblue},%
   commentstyle=\color{darkgreen},%
   stringstyle=\color{steelblue},%
   directivestyle=\color{firebrickred}%
}
%
%\parindent0ex
\renewcommand{\topfraction}{0.8}
\renewcommand{\bottomfraction}{0.8}
\renewcommand{\textfraction}{0.2}
\renewcommand{\floatpagefraction}{0.75}
\setcounter{tocdepth}{2}
\setcounter{secnumdepth}{2}
\renewcommand{\labelitemi}{$\blacktriangleright$}
%
\newcommand{\eg}{{e.\,g.}\xspace}
\newcommand{\ie}{{i.\,e.,}\xspace}
\newcommand{\zimpl}{{\sc Zimpl}\xspace}
\newcommand{\lp}{{\sc lp}\xspace}
\newcommand{\ip}{{\sc ip}\xspace}
\newcommand{\cpu}{{\sc cpu}\xspace}
\newcommand{\mip}{{\sc mip}\xspace}
\newcommand{\tsp}{{\sc tsp}\xspace}
\newcommand{\mps}{{\sc mps}\xspace}
\newcommand{\lpf}{{\sc lp}\xspace}
\newcommand{\ibm}{{\sc ibm}\xspace}
\newcommand{\zpl}{{\sc zpl}\xspace}
\newcommand{\ampl}{{\sc ampl}\xspace}
\newcommand{\ilog}{{\sc ilog}\xspace}
\newcommand{\cplex}{{\sc cplex}\xspace}
\newcommand{\scip}{{\sc scip}\xspace}
\newcommand{\sos}{{\sc sos}\xspace}

% \newcommand{\code}[1]{{\tt #1}\xspace}
%% 夹在中文里的行间代码用等宽字符显示有的时候会出现看不起的情况,我尝试了好几
%% 种办法。最后发现最好用的方法是:
%%
%% 1. 行间代码前后都加上 \xspace 与前后文拉开距离;
%% 2. 细长的行间代码字符比如 \code{[}、\code{|} 和 \code{:},翻译的时候前后加
%%    上中文引号来改善显示。
%
% \newcommand\mystrut{\rule[-1pt]{0pt}{.8em}}
% \usepackage{tcolorbox} % To support inline code.
% \newtcbox{\codetc}{on line, boxrule=0pt, boxsep=0pt, top=2pt,
% left=2pt, bottom=2pt, right=2pt, colback=magenta!7.5, colframe=white,
% fontupper={\ttfamily\mystrut}}
% \newcommand{\code}[1]{\codetc{\textcolor{magenta}{\tt #1}\xspace}}
\newcommand{\code}[1]{\xspace{\tt #1}\xspace}

\newcommand{\NN}{\ensuremath{\mathbb{N}}}
\newcommand{\NNZ}{\ensuremath{\mathbb{N}_0}}
\newcommand{\ZZ}{\ensuremath{\mathbb{Z}}}
\newcommand{\BB}{\ensuremath{\{0,1\}}}
\newcommand{\fa}{\ensuremath{\text{for all }}}
\newcommand{\argmin}{\ensuremath{\operatorname{argmin}}\xspace}
\newcommand{\argmax}{\ensuremath{\operatorname{argmax}}\xspace}
%
\headheight5mm
%\renewcommand{\footrulewidth}{\headrulewidth}
\lhead{\zimpl}
\chead{}
\rhead{}
\cfoot{\thepage}
\pagestyle{fancy}
%
\begin{document}
%\ZTPAuthor{Thorsten Koch}
%\ZTPTitle{\zimpl User Guide}
%\ZTPInfo{Best preprint of the year 2000}
%\ZTPNumber{01-20}
%\ZTPMonth{August}
%\ZTPPreprint
%\ZTPYear{2001}
%\zibtitlepage

\title{
%\vspace*{-3cm}\epsfig{file=ziblogo2.eps,width=3cm}\\[\bigskipamount]
\LARGE\zimpl 用户指南\\
\normalsize (祖萨研究院数学规划建模设计语言)\\
}
\author{托尔斯滕·科赫}
\date{\small 版本 3.7.0\\2024年10月}
\maketitle
%
\tableofcontents
\newpage
\begin{abstract}
  \zimpl 是一种轻量化的特定领域语言 (little language),用于将问题的数学模型
  描述转译为线性或 (混合) 整数规划程序,并保存为 (希望是) 能被\lp 或\mip 的
  求解器求解的\lpf 或\mps 文件格式。
  % \zimpl is a little language in order to translate the mathematical
  % model of a problem into a linear or (mixed-)integer mathematical
  % program, expressed in \lpf or \mps file format which can be read and
  % (hopefully) solved by a \lp or \mip solver.
\end{abstract}
% -----------------------------------------------------------------------------
% --- Introduction
% -----------------------------------------------------------------------------


\section{序言}
% \section{Preface}
\begin{flushright}
{\em 愿源码与你同在,卢克!}\footnote{\textbf{译者注}:“愿原力与你同在” 
(May the force be with you.),是《星球大战》系列影视作品里一句著名台词,
最初是影片故事里对拥有原力者的一种祝福和祈祷。这里原文作者利用了谐音,将“原力”
(force) 替换为了“源码” (source) 以祝福支持开源事业的读者。}
% {\em May the source be with you, Luke!}
\end{flushright}
许多\zimpl 中的功能 (以及更多它不具备的功能) 都可以在罗伯特·富勒、
大卫·N·盖伊和布莱恩·W·克宁翰合著的关于\ampl 建模语言的优秀书籍
\cite{FourierGayKernighan2003} 中找到。如果您对当前 (商用) 建模语言的最新
进展感兴趣,也可以参考文献\cite{Kallrath2004}。
% Many of the things in \zimpl (and a lot more) can be found in
% the excellent book about the modeling language \ampl
% from Robert Fourer, David N. Gay and Brian W. Kernighan
% \cite{FourierGayKernighan2003}. Those interested in an overview of the
% current state-of-the-art in (commercial) modeling languages might have
% a look at \cite{Kallrath2004}.

但另一方面,拥有程序的源代码可能带来许多优势。例如,能够在不同的架构和操作
系统上运行,能够根据需求对程序进行修改,以及不必与许可证管理器纠缠的便利性
,都可能使一个功能弱得多的程序成为更好的选择。正因如此,\zimpl 应运而生。
% On the other hand, having the source code of a program has its
% advantages. The possibility to run it regardless of architecture and
% operating system, the ability to modify it to suit the needs, and not
% having to hassle with license managers may make a much less powerful
% program the better choice.  And so \zimpl came into being.

\bigskip
迄今为止\zimpl 被逐步完善并成熟,已被应用于数个工业项目和高校教育课程,
展示出其不仅在应对大规模数学模型时,也在面向学生教育中,具备出众的能力。
而这也离不开我的早期用户阿明·菲根舒 (Armin F\"ugenschuh),马克·普费奇 
(Marc Pfetsch),萨沙·卢卡茨 (Sascha Lukac),丹尼尔·容格拉斯 (Daniel
Junglas),约尔格·兰鲍 (J\"orgRambau)和托比亚斯·阿赫特贝格 (Tobias
Achterberg),感谢他们提出的意见和问题反馈。特别感谢图奥莫·塔库拉 (Tuomo
Takkula) 对本手册的修订。
% By now \zimpl has grown up and matured. It has been used in several
% industrial projects and university lectures, showing that it is able to
% cope with large-scale models and also with students.
% This would not have been possible without my early adopters
% Armin F\"ugenschuh, Marc Pfetsch, Sascha Lukac, Daniel Junglas, J\"org
% Rambau, and Tobias Achterberg. Thanks for their comments and bug
% reports. Special thanks to Tuomo Takkula for revising this manual.

\bigskip
\zimpl 基于第三版GNU宽通用公共许可证发布。更多关于自由软件的信息另请参见
\url{http://www.gnu.org}。\zimpl 的最新版本可以在\url{http://zimpl.zib.de}
找到。
如果你发现了任何的程序错误,请发送电子邮件到邮箱\url{mailto:koch@zib.de},
请不要忘了附上示例来展示问题。如果有人开发了\zimpl 的功能扩展,我很乐意收
到补丁,并将这些改进纳入主发行版本。
% \zimpl is licensed under the GNU Lesser General Public License Version 3.
% For more information on free software see \url{http://www.gnu.org}.
% The latest version of \zimpl can be found at
% \url{http://zimpl.zib.de}.
% If you find any bugs, then please send an email to
% \url{mailto:koch@zib.de}, and do not forget to
% include an example that shows the problem.
% If somebody extends \zimpl, then I am interested in getting patches
% to include them into the main distribution.

\bigskip
% \noindent{The best way to refer to \zimpl in a publication is to cite my
%   PhD thesis \cite{Koch2004}}
\noindent{在出版物中引用\zimpl 的最佳方式是引用我的博士论文
    \cite{Koch2004}}
{\small
\begin{verbatim}
 @PHDTHESIS{Koch2004,
   author      = "Thorsten Koch",
   title       = "Rapid Mathematical Programming",
   school      = "Technische {Universit\"at} Berlin",
   year        = "2004",
   url         = "http://www.zib.de/Publications/abstracts/ZR-04-58/",
   note        = "ZIB-Report 04-58"
}
\end{verbatim}

%--------------------------------------------------------------------------------------
% TODO: Bei den SET examples koennte man parallel das
%       Ergenis zeigen:
%       set A := { 1..4}     |  { 1, 2, 3, 4 }
%
%       min/max expr beispiel
%
%\section{Introduction}


%% Consider the following linear program:
%% $$
%% \begin{array}{rll}
%% \min& 2 x + 3 y\\
%% \mbox{subject to}& x + y& \leq 6\\
%% &x,y&\ge 0\\
%% \end{array}
%% $$
%% The standard format used to feed such a problem into a solver
%% is called \mps.
%% \ibm invented it for the Mathematical
%% Programming System/360 \cite{Kallrath2004b,Spielberg2004} in the sixties.
%% Nearly all available \lp and \mip solvers can read this format.
%% While \mps is a nice format to punch into a punch card and at least a
%% reasonable format to read for a computer, it is quite unreadable
%% for humans. For instance, the \mps file of the above linear program
%% looks as follows:

%% \medskip
%% \lstset{language=mps,%
%% basicstyle=\sffamily\footnotesize,%
%% numberstyle=\sffamily\tiny\color{siennabrown},stepnumber=1}
%% \begin{lstlisting}[frame=]{}
%%     NAME        ex1.mps
%%     ROWS
%%      N  OBJECTIV
%%      L  c1
%%     COLUMNS
%%         x         OBJECTIV             2
%%         x         c1                   1
%%         y         OBJECTIV             3
%%         y         c1                   1
%%     RHS
%%         RHS       c1                   6
%%     BOUNDS
%%      LO BND       x                    0
%%      LO BND       y                    0
%%     ENDATA
%% \end{lstlisting}

%% \bigskip
%% \noindent Another possibility is the \lpf format \cite{CPlex80}, which is more
%% readable\footnote{
%% The \lpf format has also some idiosyncratic restrictions. For example
%% variables should not be named \code{e12} or the like. And it is not
%% possible to specify ranged constraints.}
%% but is only supported by a few solvers.
%% {
%% \small
%% \begin{verbatim}
%%    Minimize
%%     cost:  +2 x +3 y
%%    Subject to
%%     c1:  +1 x +1 y <= 6
%%    End
%% \end{verbatim}
%% }
%% \noindent But since each coefficient of the matrix $A$ must be stated
%% explicitly it is also not a desirable choice to develop a mathematical
%% model.

%% \medskip
%% \noindent Now, with \zimpl it is possible to write this:
%% {\small
%% \begin{verbatim}
%%    var x;
%%    var y;
%%    minimize cost: 2 * x + 3 * y;
%%    subto c1: x + y <= 6;
%% \end{verbatim}
%% }
%% \noindent and have it automatically translated into \mps or \lpf format.
%% While this looks not much different from what is in the \lpf format,
%% the difference can be seen, if we use indexed variables.
%% Here is an example. This is the \lp:
%% $$
%% \begin{array}{rl}
%% \min& 2 x_1 + 3 x_2 + 1.5 x_3\\
%% \mbox{subject to}&\sum^3_{i=1} x_i \leq 6\\
%% &x_i\ge 0\\
%% \end{array}
%% $$
%% And this is how to tell it to \zimpl:

%% \medskip
%% \lstset{language=zimpl,%
%% basicstyle=\sffamily\footnotesize,%
%% numberstyle=\sffamily\tiny\color{siennabrown},stepnumber=1}
%% \begin{lstlisting}[frame=]{}
%%    set I      := { 1 to 3 };
%%    param c[I] := <1> 2, <2> 3, <3> 1.5;
%%    var   x[I] >= 0;
%%    minimize cost: sum <i> in I : c[i] * x[i];
%%    subto    cons: sum <i> in I : x[i] <= 6;
%% \end{lstlisting}



\clearpage

\section{简介}
% \section{Introduction}

考虑一个$s-t$最短路问题的线性规划形式,针对有向图$(V, A)$,边的成本系数
记为 $c_{ij}$,对于集合$A$中的所有边 $(i, j) \in A$,建模如下:
% Consider the \lpf formulation of the shortest $s,t$-path problem,
% applied to some directed graph $(V,A)$ with cost coefficient $c_{ij}$
% for all $(i,j) \in A$:

\begin{equation}
  \begin{array}{rll}
    \min& \displaystyle\sum_{(i,j) \in A} c_{ij} x_{ij}\\
    & \displaystyle\sum_{(iv)\in \delta^-(v)} x_{iv} = \sum_{(vi)\in
      \delta^+(v)} x_{vi}
       \mbox{\quad for all }v \in V\setminus\{s,t\} \\
       &  \\
    & x_{ij} \in \{0,1\}, \mbox{for all {i,j} in A}\\
  \end{array}\label{eq:shortestpath}
\end{equation}
%
其中,对于任意$v \in V$,定义$\delta^+(v) := {(v, i) \in A}$和
$\delta^-(v) := {(i, v) \in A}$。对于一个特定的图,模型可以具象表述为:
% where $\delta^+(v):=\{(v,i) \in A\}$, $\delta^-(v):=\{(i,v) \in A\}$
% for $v\in V$. For a given graph  the instantiation is

\begin{minipage}[c]{0.3\linewidth}
  \begin{center}
    \label{stexample}
    \includegraphics[height=4cm]{stexample}
  \end{center}
\end{minipage}
%
\begin{minipage}[c]{0.5\linewidth}
  $$
   \begin{array}{rll}
    \min& 17 x_{sa} + 47 x_{sb} + 19 x_{ab} + 53 x_{at} + 23 x_{bt}     \\
    \mbox{subject to}& x_{sa} = x_{ab} + x_{at} \\
    & x_{sb} + x_{ab} = x_{bt} \\
    &x_{ij} \in \{0,1\}, \mbox{for all {i,j}}\\
  \end{array}
  $$
\end{minipage}

现将此类问题输入求解器所使用的标准格式称为\mps,这一格式系\ibm 为20世纪60
年代的数学规划系统 System/360 设计的~\cite{Kallrath2004b,Spielberg2004}。
尽管几乎所有现存的线性规划 (\lp) 和混合整数规划 (\mip) 求解器都能读取这种
格式,虽然\mps 格式非常适合通过穿孔纸带输入,且对计算机来说也至少是可读的
,但对人类而言,却几乎不可理解。例如上述线性规划问题的\mps 文件内容如下:
% The standard format used to feed such a problem into a solver is
% called \mps which was invented by \ibm for the Mathematical
% Programming System/360 in the
% sixties~\cite{Kallrath2004b,Spielberg2004}.  Nearly all available \lp
% and \mip solvers can read this format.  While \mps is a nice format to
% punch into a punch card and at least a reasonable format to read for a
% computer, it is quite unreadable for humans. For instance, the \mps
% file of the above linear program looks as follows:

\lstset{language=mps,%
basicstyle=\sffamily\footnotesize,%
numberstyle=\sffamily\tiny\color{siennabrown},stepnumber=1}
\begin{lstlisting}[frame=]{}
NAME          shortestpath.lp
ROWS
 N  Obj
 E  c0
 E  c1
 E  c2
COLUMNS
    INTSTART  'MARKER'       'INTORG'
    x0        Obj        17  c0            1
    x0        c2          1
    x1        Obj        47  c1            1
    x1        c2          1
    x2        c1          1  c0           -1
    x2        Obj        19
    x3        Obj        53  c0           -1
    x4        c1         -1  Obj          23
RHS
    RHS       c2          1
BOUNDS
 UP Bound     x0          1
 UP Bound     x1          1
 UP Bound     x2          1
 UP Bound     x3          1
 UP Bound     x4          1
ENDATA
\end{lstlisting}

\bigskip
\noindent 而另一个可能的格式是\lpf 格式\cite{CPlex80}相比之下具有更高的可
读性\footnote{\lpf 格式也具有一些乖张的限制。比如变量不能被命名为
  \code{e12}或者类似的名称,而且也不能指定范围约束。},很接近具象化的表述
  ,但是只被极少数的求解器所支持。
% \noindent Another possibility is the \lpf format \cite{CPlex80} which
% is more readable\footnote{ The \lpf format has also some idiosyncratic
%   restrictions. For example variables should not be named \code{e12}
%   or the like. and it is not possible to specify ranged
%   constraints.}, quite similar to the instantiation,
% but only supported by a few solvers:
{ \small
\begin{verbatim}
Minimize
 Obj: +17 x0 +47 x1 +19 x2 +53 x3 +23 x4
Subject to
 c0: -1 x3 -1 x2 +1 x0 = +0
 c1: -1 x4 +1 x2 +1 x1 = +0
 c2: +1 x1 +1 x0 = +1
Bounds
 0 <= x0 <= 1
 0 <= x1 <= 1
 0 <= x2 <= 1
 0 <= x3 <= 1
 0 <= x4 <= 1
Generals
 x0 x1 x2 x3 x4
End
\end{verbatim}
}
\noindent 而鉴于其中又必须精确指定矩阵~$A$的每个参数,这仍算不上是一种数学
建模语言的理想选择。
% \noindent Since each coefficient of the matrix~$A$ must be stated
% explicitly it is also not a desirable choice for the development of a
% mathematical model.

\subsubsection{抽象公式表述}
% \subsubsection{Abstract formulation}

而现在,这一模型在\zimpl 中可以写作:
% Now, with \zimpl it is possible to write

{\small
\begin{verbatim}
set V      :={"a","b","s","t"};
set A      :={<"s","a">, <"s","b">, <"a","b">, <"a","t">, <"b","t">};
param c[A] := <"s","a"> 17, <"s","b"> 47, <"a","b"> 19, <"a","t"> 53,
              <"b","t"> 23;
defset dminus(v) := {<i,v> in A};
defset dplus(v)  := {<v,j> in A};
var x[A] binary;
minimize cost: sum<i,j> in A: c[i,j] * x[i,j];
subto fc:
      forall <v> in V - {"s","t"}:
        sum<i,v> in dminus(v): x[i,v] == sum<v,i> in dplus(v): x[v,i];
subto uf:
      sum<s,i> in dplus("s"): x[s,i] == 1;
\end{verbatim}
}

\noindent ——请将这段代码与~(\ref{eq:shortestpath}) 相比较。将这段代码输
入\zimpl 即可自动生成\mps 或\lpf 文件。
% \noindent -- compare this with~(\ref{eq:shortestpath}). Feeding the
% script into \zimpl will automatically generate \mps or \lpf files.


\medskip

像\zimpl 这样的建模语言的价值在于它们具备直接处理数学模型本身的能力,而不
是仅仅对系数进行处理。除此之外,模型的具象 (可以理解为是生成的\lpf 文件或
\mps 文件) 通常会通过外部数据生成。从某种意义上来讲,“具象化”正是将模型
应用于外部数据的结果。在我们上面的算例中,所谓外部数据就是带有成本系数和指
定的$s$和$t$的图,而模型则是$st$最短路优化问题的数学公式。当然,\zimpl 也
支持通过文件来初始化模型。例如,上述这个\zimpl 脚本的第一行也可以写作:%
% The value of modeling languages like \zimpl lies in their ability to
% work directly on the mathematical model in lieu of dealing merely with
% coefficients. Furthermore, more often than not instantiations (read
% \lpf or \mps-files) of models are generated from some external
% data. In some sense, these are the result of the model applied to this
% external data. In our example above, the external data is the graph
% together with the cost coefficients and the specification of $s$ and
% $t$, and the model is the formulation of the shortest $st$-path
% problem. Of course, \zimpl supports the initialization of models
% from files. For instance, the first lines from the \zimpl script above
% can be replaced by {\small
{\small
\begin{verbatim}
set       V:= {read "nodes.txt" as "<1s>"};
set       A:= {read "arcs.txt"  as "<1s,2s>"};
param c[A] := read "arcs.txt"  as "<1s,2s>3n";
\end{verbatim}
} 而\zimpl 也可以根据“nodes.txt”和“arcs.txt”两个文件中的定义生成任何一
个最短路问题的实例。这些文件中具体的格式规定将在\ref{initfromfile}节中叙述。
% } and the \zimpl script is ready produce instances of any shortest
% path problem defined by the files ``nodes.txt'' and ``arcs.txt''. The
% format of those files will be described in Subsection~\ref{initfromfile}.

\section{命令行调用}
% \section{Invocation}

要对文件\code{ex1.zpl}中给定的模型运行\zimpl 程序,需要键入如下命令:
% In order to run \zimpl on a model given in the file \code{ex1.zpl} type the command:
\begin{verbatim}
   zimpl ex1.zpl
\end{verbatim}
命令的一般形式为:
% In general terms the command is:
\begin{verbatim}
   zimpl [options] <input-files>
\end{verbatim}
命令可以接受多个文件输入,按顺序读取,如同被合并为一个单一的大文件。如果在
处理过程中产生任何报错,\zimpl 会打印错误信息并中止运行。若一切正常,结果
将根据指定的选项被写入至少两个文件中。
% It is possible to give more than one input file. They are read one
% after the other as if they were all one big file.
% If any error occurs while processing, \zimpl prints out an
% error message and aborts. In case everything goes well, the results
% are written into two or more files, depending on the specified options.

第一个文件是根据模型生成的\cplex、\lp 格式,\mps 格式或“人可读”格式的优
化问题文件,后缀名分别是 \emph{.lp},\emph{.mps},或者\emph{.hum}。
而另一个是\emph{table} (表格) 文件,以后缀名\emph{.tbl}结尾。表格文件列出
了模型中使用到的所有变量及约束的名称,以及它们在优化问题文件中对应的名称。
之所以会造成这一名称转译的原因在于\mps 文件格式中的名称长度限制为8个字符。
而且在\lp 文件中也同样限制名称的长度。具体限制取决于所使用的版本。
\cplex~0.7 中的限制为16字符,且会无视名称中其余的部分,而\cplex~0.9 则上限
为255字符,但部分命令的输出中只会展示前20个字符。
% The first output file is the problem generated from the model in either
% \cplex \lp, \mps, or  a ``human readable'' format,
% with extensions \emph{.lp}, \emph{.mps}, or \emph{.hum}, respectively.
% The next one is the \emph{table} file which has the extension \emph{.tbl}.
% The table file lists all variable and constraint names used in the model
% and their corresponding names in the problem file.
% The reason for this name translation is the limitation of the length
% of names in the \mps format to
% eight characters. Also the \lp format
% restricts the length of names. The precise limit is depending on the
% version. \cplex~7.0 has a limit of 16 characters, and ignores
% silently the rest of the name, while \cplex~9.0 has a limit of 255
% characters, but will for some commands only show the first 20 characters
% in the output.

% 第三个文件是可选的\cplex 分支顺序文件。
%The third file is and optional \cplex branching order file.

\zimpl 可解析的完整参数列表详见表\ref{tab:zimpl-options}。
一个典型的\zimpl 调用命令如下例所示:%
% A complete list of all options understood by \zimpl can be found in
% Table~\ref{tab:zimpl-options}.
% A typical invocation of \zimpl is for example:
\begin{verbatim}
   zimpl -o solveme -t mps data.zpl model.zpl
\end{verbatim}
这将读取文件\code{data.zpl}和\code{model.zpl}作为输入,并生成输出文件
\code{solveme.mps}和\code{solveme.tbl}。需要注意的是,如果指定输出为\mps 
格式且优化目标为极大化,目标函数的正负号将被反转。这是因为\mps 文件格式不
支持直接指定目标函数的优化方向,其默认假设为极小化。
% This reads the files data.zpl and model.zpl as
% input and produces as output the files solveme.mps and solveme.tbl.
% Note that in case \mps-output is specified for a maximization problem,
% the objective function will be inverted, because the \mps format has no
% provision for stating the sense of the objective function. The default
% is to assume minimization.

\begin{table}[hbtp]
{\sffamily\small\centering
\begin{tabular}{lp{130mm}}
% \begin{tabular}{lp{104mm}}
\toprule
-t \emph{format} & 指定输出的格式。可以是默认的\code{lp}格式,或\code{mps}
                  格式,或者仅供人类阅读的\code{hum}格式。还可以是
                  \code{rlp}格式,此格式与\code{lp}相同,但基于\emph{seed}
                  参数提供的随机种子,对行和列进行了随机置换。另一种可能的
                  输出格式是\code{pip}格式,用于描述多项式整数规划 (
                  Polynomial IP) 问题;此外还有\code{q}\emph{x}格式,用于
                  描述二次无约束0-1优化 (QUBO,即Quadratic Unconstrained 
                  Binary Optimization) 问题,其中\emph{x}为格式选项:
                  \emph{0}表示使用从零开始的矩阵索引(默认是从1开始),
                  \emph{c}表示在文件中使用字符\code{c}作为注释行指示符(默
                  认是\code{\#}),\emph{p}表示在实例文件的首行写入字符
                  \code{p}。 \\
% -t \emph{format} & Selects the output format. Can be either \code{lp},
%                   which is default, or \code{mps}, 
%                   or \code{hum}, which is only human readable,
%                   or \code{rlp}, which is the
%                   same as \code{lp} but with rows and columns randomly
%                   permuted. The permutation is depending on the
%                   \emph{seed}. Also possible is \code{pip} which
%                   means Polynomial IP, or \code{q}\emph{x} which
%                   means QUBO. \emph{x} are format
%                   options. Available are: \emph{0} for zero based matrix
%                   indexing (1 is default), \emph{c} for using 'c' as
%                   comment line indicator in the file (default is '\#'),
%                   and \emph{p} for writing an 'p' in the first line of
%                   the instance. \\
-o \emph{name}   & 选择输出文件的文件名不含扩展名。 \\
                & 默认为输入的第一个文件的文件名,不含路径和扩展名。\\
% -o \emph{name}   & Sets the base-name for the output files.\\
%                 & Defaults to the name of the first input file with
%                   its path and extension stripped off.\\
-F \emph{filter} & 将输出通过管道传递给一个过滤器。字符串中的\%s将被替换
                  为输出文件的文件名。举个例子:参数
                  \code{-F "gzip -c >\%s.gz"}
                  可以压缩所有输出的文件。 \\
% -F \emph{filter} & The output is piped through a filter. A \%s in the
%                   string is replaced by the output filename. For example
%                   \code{-F "gzip -c >\%s.gz"} would compress all the
%                   output files.\\
-l \emph{length} & 设置\code{lp}文件格式中变量和约束的最大长度到
                  \emph{length}。 \\
% -l \emph{length} & Sets maximal length of variables and
%                   constraints for \code{lp} format files to \emph{length}.\\
-n \emph{cform}  & 选择生成约束名称的格式。如果设置为\code{cm},则约束将以
                  字符`c'开头,并被编号为$1\ldots n$。设置为\code{cn}时,
                  约束名称将使用\code{subto}语句中指定的名称,并在该语句内
                  部编号为$1\ldots n$。设置为\code{cf}时,约束名称将以
                  \code{subto}中指定的名称开头,随后加上编号$1\ldots n$ 
                  (类似于\code{cm}),并附加来自\code{forall}语句的所有局部
                  变量。\\
% -n \emph{cform}  & Select the format for the generation of constraint
%                   names. Can be \code{cm} which will number them
%                   $1\ldots n$ with a `c' in front. \code{cn} will use
%                   the name supplied in the \code{subto} statement and
%                   number them $1\ldots n$ within the statement.
%                   \code{cf} will use the name given with the \code{subto},
%                   then a $1\ldots n$ number like in \code{cm} and then
%                   append all the local variables from the forall statements.\\
-P \emph{filter} & 将输入通过管道传递给一个过滤器。字符串中的\%s将被替换为
                  输入文件的文件名。举个例子:参数
                  \code{-P "cpp -DWITH\_C1 \%s"}
                  可将输入的文件传递给C语言的预处理器对输入文件进行处理 \\
% -P \emph{filter} & The input is piped through a filter. A \%s in the
%                   string is replaced by the input filename. For example
%                   \code{-P "cpp -DWITH\_C1 \%s"} would pass the input
%                   file through the C-preprocessor.\\
-s \emph{seed}   & 用于随机数生成器的一个正值的随机种子\code{seed}。例如
                 \code{-s `date +\%N'}
                 可提供不断变更的随机种子。 \\
% -s \emph{seed}   & Positive \code{seed} number for the random number generator.
%                  For example, \code{-s `date +\%N`} will produce changing
%                  random numbers.\\
-v \emph{0..5}   & 设置输出日志的详细等级。0表述静默,1为默认水平。2为详细
                  输出,3和4为细致输出,5为调试级信息。\\
% -v \emph{0..5}   & Set the verbosity level. 0 is quiet, 1 is default,
%                   2 is verbose, 3 and 4 are chatter, and 5 is debug.\\
-D \emph{name=val} & 设置参数\emph{name}为指定值。这相当于在代码开头增加了
                    \code{param name:=val}
                    一行内容。如果\zimpl 文件中已经声明了同名的参数而
                    \code{-D}选项又同样设置了相同的名称,则以\code{-D}的指
                    定为准。\\
% -D \emph{name=val} & Sets the parameter \emph{name} to the specified
%                   value. This is equivalent to having this line in the
%                   \zimpl program: \code{param name:=val}. If there is
%                   a declaration in the \zimpl file and a \code{-D}
%                   setting for the same name, the latter takes precedent.\\
%\hline
-b & 启用bison语法解析器的输出。\\
% -b & Enables bison debug output.\\
-f & 启用flex词法分析器的输出。\\
% -f & Enables flex debug output.\\
-h & 显示帮助信息。\\
% -h & Prints a help message.\\
-m & 生成一份\cplex \code{mst} (Mip STart) 文件\\
% -m & Writes a \cplex \code{mst} (Mip STart) file.\\
-O & 尝试通过预处理来简化生成的线性规划模型。\\
% -O & Tries to reduce the generated LP by doing some presolve analysis.\\
-r & 生成一份\cplex \code{ord}分支次序文件。\\
% -r & Writes a \cplex \code{ord} branching order file.\\
-V & 显示版本号。\\
% -V & Prints the version number.\\
\bottomrule
\end{tabular}
}
\caption{\zimpl 参数选项}%
\label{tab:zimpl-options}
\end{table}

% -----------------------------------------------------------------------------
% --- Format
% -----------------------------------------------------------------------------

% \section{Format}
\section{语法规范}

每个\zpl 文件包含六种类型的声明语句:
% Each \zpl-file consists of six types of statements:
\begin{itemize}
\setlength{\itemsep}{0pt}%
\item 集合 Sets
\item 参数 Parameters
\item 变量 Variables
\item 目标 Objective
\item 约束 Constraints
\item 函数定义 Function definitions
\end{itemize}
%
每一句语句都以一个分号结尾。
除了字符串以外,“\code{\#}”符号之后到这行末尾的所有内容都会被视作注释忽略。
如果一行以单词\code{include}开头,后跟有带双引号的文件名,则会读取并处理该
文件,而不是该行。
% Each statement ends with a semicolon.
% Everything from a hash-sign \code{\#}, provided it is not part of a string, to
% the end of the line is treated as a
% comment and is ignored.
% If a line starts with the word \code{include} followed by a filename in double
% quotation marks, then this file is read and processed instead of the line.

%-----------------------------------------------------------------------------------------
% \subsection{Expressions}
\subsection{表达式}
%-----------------------------------------------------------------------------------------
\zimpl 基于两种最基本的数据类型:字符串和数值。
凡是需要提供数字或字符串的地方,也可以使用对应值类型的参数。在大多数情况下
,可以使用表达式作为值,而不仅仅是写一个数字或字符串。
运算优先级通常取决于一般规定,但可以使用括号来显式指定求值顺序。
% \zimpl works on its lowest level with two types of data: Strings and
% numbers.
% Wherever a number or string is required it is also possible to use a
% parameter of the corresponding value type. In most cases, expressions are
% allowed instead of just a number or a string.
% The precedence of operators is the usual one, but
% parentheses can always be used to specify the evaluation order explicitly.
%如果有疑问就使用括号。
%If in doubt use parenthesis.

% \subsubsection{Numeric expressions}
\subsubsection{数值表达式}
\zimpl 中的数字可以通过一般的写法给定,如2,-6.5或5.23e-12。
数值表达式的形式包括数字、具有数值类型值的参数,以及
表\ref{tab:zimpl-functions}列出的任何一种运算符或函数。除此之外
表\ref{tab:zimpl-double}中所示的函数也是可以使用的。需要注意的是,这些函数
仅使用普通的双精度浮点运算进行计算,因此精度有限。关于如何使用
\code{max}和\code{min}函数的例子,可以在第\pageref{ssec:parameters}页的
第\ref{ssec:parameters}节中找到
\footnote{\textbf{译者注}:经译者实测,表\ref{tab:zimpl-functions}所列表达
  式除加减乘除等运算符外,在\zimpl 中使用似乎均会报错,正确用法形如
  \code{sum <i> in I: e(i)}在\ref{ssec:parameters}节中展示。其余函数亦同。
  为考证这一问题,译者已向\zimpl 的Git仓库提交了Issue,详情参考
  \url{https://github.com/scipopt/zimpl/issues/2}\label{fn:translator_1}。
  该Issue于1月2日得到了开发者的回复,其中\code{max}和\code{min}函数的形式
  已经得到了修复,但\code{sum}和\code{prod}仍然存在问题。
}。
% A number in \zimpl can be given in the usual format, \eg as 2, -6.5 or 5.234e-12.
% Numeric expressions consist of numbers, numerically valued parameters, and
% any of the operators and functions listed in Table~\ref{tab:zimpl-functions}.
% Additionally the functions shown in Table~\ref{tab:zimpl-double} can be
% used. Note that those functions are only computed with normal double precision
% floating-point arithmetic and therefore have limited
% accuracy. Examples on how to use the \code{min} and \code{max}
% functions can be found in Section~\ref{ssec:parameters} on page~\pageref{ssec:parameters}.

\begin{table}[htbp]
\centering
{\sffamily\small
\begin{tabular}{lll}
\toprule
\code{a${}^\wedge$b}, \code{a**b} &$a$的$b$次方              & $a^b$,$b$必须为整数\\
% \code{a${}^\wedge$b}, \code{a**b} &$a$ to the power of $b$   & $a^b$, $b$ must be integer\\

\code{a+b}                       &加法                       & $a+b$\\
% \code{a+b}                       &addition                  & $a+b$\\

\code{a-b}                       &减法                       & $a-b$\\
% \code{a-b}                       &subtraction               & $a-b$\\

\code{a*b}                       &乘法                       & $a\cdot b$\\
% \code{a*b}                       &multiplication            & $a\cdot b$\\

\code{a/b}                       &除法                       & $a/b$\\
% \code{a/b}                       &division                  & $a/b$\\

\code{a mod b}                   &取模                       & $a\mod b$\\
% \code{a mod b}                   &modulo                    & $a\mod b$\\

%\code{a div b}                   &整数除法 & \\
%\code{a div b}                   &integer division           & \\

\code{abs(a)}                    &绝对值                     & $|a|$\\
% \code{abs(a)}                    &absolute value            & $|a|$\\

\code{sgn(a)}                    &符号函数                   & 
$x>0\Rightarrow 1, x<0\Rightarrow -1,\text{否则为}0$\\
% \code{sgn(a)}                    &sign                      &
% $x>0\Rightarrow 1, x<0\Rightarrow -1,\text{else }0$\\

\code{floor(a)}                  &向下取整                   & $\lfloor a\rfloor$\\
% \code{floor(a)}                  &round down                & $\lfloor a\rfloor$\\

\code{ceil(a)}                   &向上取整                   & $\lceil a\rceil$\\
% \code{ceil(a)}                   &round up                  & $\lceil a\rceil$\\

\code{round(a)}                  &四舍五入                   & $\lfloor a \rceil$\\
% \code{round(a)}                  &round towards zero        & $\lfloor a \rceil$\\

\code{a!}                        &阶乘                       & $a!$,
$a$ 必须为非负整数\\
% \code{a!}                        &factorial                 & $a!$,
% $a$ must be nonnegative integer \\

\code{min(S)}                    &集合元素的最小值           &$\min_{s\in S}$\\
% \code{min(S)}                    &minimum of a set          &$\min_{s\in S}$\\

\code{min <s> in S: e(s)}          &函数在集合上取得的最小值   &$\min_{s\in S} e(s)$\\
% \code{min(s in S) e(s)}          &minimum over a set        &$\min_{s\in S} e(s)$\\

\code{max(S)}                    &集合元素的最大值           &$\max_{s\in S}$\\
% \code{max(S)}                    &maximum of a set          &$\max_{s\in S}$\\

\code{max <s> in S: e(s)}          &函数在集合上取得的最大值   &$\max_{s\in S} e(s)$\\
% \code{max(s in S) e(s)}          &maximum over a set        &$\max_{s\in S} e(s)$\\

\code{min(a,b,c,\ldots,n)}       &列表元素的最小值           &$\min (a,b,c,\ldots,n)$\\
% \code{min(a,b,c,\ldots,n)}       &minimum of a list         &$\min (a,b,c,\ldots,n)$\\

\code{max(a,b,c,\ldots,n)}       &列表元素的最大值           &$\max (a,b,c,\ldots,n)$\\
% \code{max(a,b,c,\ldots,n)}       &maximum of a list         &$\max (a,b,c,\ldots,n)$\\

\code{sum(s in S) e(s)}          &集合元素代入函数计算后求和 &$\sum_{s\in S} e(s)$\\
% \code{sum(s in S) e(s)}          &sum over a set            &$\sum_{s\in S} e(s)$\\

\code{prod(s in S) e(s)}         &集合元素代入函数计算后乘积 &$\prod_{s\in S} e(s)$\\
% \code{prod(s in S) e(s)}         &product over a set        &$\prod_{s\in S} e(s)$\\

\code{card(S)}                   &集合的技术                 &$|S|$\\
% \code{card(S)}                   &cardinality of a set      &$|S|$\\

\code{random(m,n)}               &伪随机数                   &$\in[m,n]$, rational \\
% \code{random(m,n)}               & pseudo random number     &$\in[m,n]$, rational \\

\code{ord(A,n,c)}                &序数                      &集合$A$中第n个元素的第c个成分\\
% \code{ord(A,n,c)}                &ordinal                   &c-th component of the n-th\\
%                                  &                          & element of set $A$.\\

\code{length(s)}                 &字符串的长度              &字符串$s$的字符个数\\
% \code{length(s)}                 &length of a string        &number of characters in $s$\\

\code{if a then b}               &                          &\\
% \code{if a then b}               &                          &\\

\code{else c end}        &\raisebox{1ex}[0cm][0cm]{条件判断}
   &\raisebox{1ex}[0cm][0cm]{$\left\{\begin{array}{rl}b,&\text{if }
   a=\text{true}\\c,&\text{if } a=\text{false}\end{array}\right.$}\\\\
% \code{else c end}        &\raisebox{1ex}[0cm][0cm]{conditional}
%    &\raisebox{1ex}[0cm][0cm]{$\left\{\begin{array}{rl}b,&\text{if }
%    a=\text{true}\\c,&\text{if } a=\text{false}\end{array}\right.$}\\\\

\bottomrule
\end{tabular}
}
\caption{有理算术函数}%
% \caption{Rational arithmetic functions}
\label{tab:zimpl-functions}
\end{table}

%With $\min$ and $\max$ it is possible to find the minimum/maximum
%member of an one dimensional set of numeric values.
%\code{card} gives the cardinality of a set.

\begin{table}[hbtp]
\centering
{\sffamily\small
\begin{tabular}{lll}
\toprule
\code{sqrt(a)} &平方根         & $\sqrt a$\\
\code{log(a)}  &以10为底的对数 & $\log_{10}a$\\
\code{ln(a)}   &自然对数       & $\ln a$\\
\code{exp(a)}  &指数函数       & $e^a$\\
%\code{random(a,b)}               &$\left[a,b\right]$ 范围内的随机数 &\\

% \code{sqrt(a)}                   &square root               & $\sqrt a$\\
% \code{log(a)}                    &logarithm to base 10      & $\log_{10}a$\\
% \code{ln(a)}                     &natural logarithm         & $\ln a$\\
% \code{exp(a)}                    &exponential function      & $e^a$\\
% %\code{random(a,b)}               &random number in $\left[a,b\right]$&\\
\bottomrule
\end{tabular}
}
\caption{双精度函数}%
% \caption{Double precision functions}
\label{tab:zimpl-double}
\end{table}

% \subsubsection{String expressions}
\subsubsection{字符串表达式}
字符串由双引号“\code{"}”包裹,形如\code{"Hello Keiken"}。两个字符串可以通过“
\code{+}”运算符拼接,例如,如果使用\code{"Hello " + "Keiken"}将会得到\code{"Hello
  Keiken"}。函数\code{substr(string, begin, length)}可用于提取字符串中的特
定部分。其中,\code{begin}是要使用的第一个字符,计数按照第一个字符从0开始。要提
取的字符串长度可通过\code{length}函数来确定。
% A string is delimited by double quotation marks \code{"},
% \eg\ \code{"Hallo Keiken"}. Two strings can be concatenated using the
% \code{+} operator, \ie\ \code{"Hallo " + "Keiken"} gives \code{"Hallo
%   Keiken"}. The function \code{substr(string, begin, length)}
% can be used to extract parts of a string. \code{begin} is the first 
% character to be used, and counting starts with zero at first character.
% If \code{begin} is negative, then counting starts
% at the end of the string. \code{length} is the number of charaters to
% extract starting at \code{begin}. The length of a string can be determined
% using the \code{length} function.


% \subsubsection{Boolean expressions}
\subsubsection{逻辑表达式}
返回值为\emph{true}或\emph{false}的表达式。对于数值和字符串,定义了关系操
作符如 $<$,$<=$,$==$,$!\!\!=$和$>=$。逻辑表达式通过\code{and},
\code{or}和\code{xor}\footnote{
  $a \text{ xor } b :=a\wedge\neg b\vee \neg a\wedge b$}连接,并可通过
\code{not}取反。
表达式 \emph{元组} \code{in} \emph{集合表达式} (将在下一章节中解释)
可用于测试元组中集合成员的关系。
逻辑表达式也可以在\emph{if}语句的\emph{then}或者\emph{else}的部分中使用。
% These evaluate either to \emph{true} or to \emph{false}. For numbers and
% strings the
% relational operators $<$, $<=$, $==$, $!\!\!=$, $>=$, and $>$ are
% defined.
% Combinations of Boolean expressions with \code{and},
% \code{or}, and
% \code{xor}\footnote{$a \text{ xor } b :=a\wedge\neg b\vee \neg a\wedge b$}
% and negation with \code{not} are possible.
% The expression \emph{tuple} \code{in} \emph{set-expression} (explained
% in the next section) can be used to test set membership of a tuple.
% Boolean expressions can also be in the \emph{then} or \emph{else} part of an
% \emph{if} expression.

% \subsubsection{Variant expressions}
\subsubsection{变量表达式}
以下的内容可能是一个数值,字符串或逻辑的表达式,取决于\emph{expression}的
部分是字符串,逻辑还是数值表达式:
% The following is either a numeric, string or boolean expression, depending on
% whether \emph{expression} is a string, boolean, or numeric expression:

\smallskip
\code{if} \emph{逻辑表达式} \code{then}
\emph{表达式} \code{else} \emph{表达式} \code{end}
% \code{if} \emph{boolean-expression} \code{then}
% \emph{expression} \code{else} \emph{expression} \code{end}

\smallskip

\noindent 同样地,\code{ord(}{}\emph{集合, 元组编号, 分量编号}\code{)}
函数的返回值是集合中某个具体的元素 (关于集合的更多细节将在后文说明)。函数
的返回值类型取决于集合中分量 (component) 的类型。如果集合中的分量是数值,
函数返回数值;如果是字符串或布尔值,则返回对应类型的值。
% \noindent The same is true for the \code{ord(}{}\emph{set,
%   tuple-number, component-number}\code{)} function which evaluates to
% a specific element of a set (details about sets are covered below).

% \subsection{Tuples and sets}
\subsection{元组与集合}

元组是具有固定维度的有序矢量,分量为数值或字符串类型。集合内可包含(有限多
个)元组。每个元组在集合中都是不重复的。在\zimpl 中所有集合都是内部有序的
,但没有特定的顺序。集合通过花括号来包裹,形如“\code{\{}”和“\code{\}}”。一个集
合中的所有元组必须具有相同个数的分量。对于一个特定集合中的所有元组,它们各
自的第$n$个元素的数据类型必须相同,也就是说它们必须全都是数值或者全都是字
符串。元组的定义通过尖括号$<$和$>$包裹,示例如\code{$<$1,2,"x"$>$}。各个分
量通过逗号分隔。如果元组是一维的,则可以在一列元素中省略元组的包裹符,但在
这种情况下,定义中的所有元组都必须省略它们。比如\code{\{1,2,3\}}是合法的定
义,而\code{\{1,2,$<$3$>$\}}是不合法的。
% A tuple is an ordered vector of fixed dimension where each component
% is either a number or a string. Sets consist of (a finite number of)
% tuples. Each tuple is unique within a set. All sets in \zimpl are
% internally ordered, but there is no particular order.  Sets are
% delimited by braces, \code{\{} and~\code{\}}, respectively.  All
% tuples of a specific set have the same number of components.  The type
% of the $n$-th component for all tuples of a set must be the same, \ie
% they have to be either all numbers or all strings.  The definition of
% a tuple is enclosed in angle brackets $<$ and $>$,
% \eg\ \code{$<$1,2,"x"$>$}. The components are separated by commas.  If
% tuples are one-dimensional, it is possible to omit the tuple
% delimiters in a list of elements, but in this case they must be
% omitted from all tuples in the definition, \eg\ \code{\{1,2,3\}} is
% valid while \code{\{1,2,$<$3$>$\}} is not.

集合可通过集合语句来定义,包括关键字\code{set},集合名称及赋值操作符
\code{:=},以及一个合法的集合表述。
% Sets can be defined with the set statement. It consists of
% the keyword \code{set}, the name of the set, an assignment operator
% \code{:=}, and a valid set expression.

集合通过使用模板元组 (template tuple) 来引用,模板元组由占位符 
(placeholders) 组成。这些占位符会被相应元组中各分量的值所替代。例如,一个
由二维元组组成的集合$S$可以通过\code{<a,b> in S}来引用。如果模板元组中的某
些占位符被赋予了实际值,那么只有那些与这些值匹配的元组会被选取 (extracted) 
。例如,\code{<1,b> in S}只会选取第一个分量为“\code{1}”的元组。需要注意的是,
如果占位符的名称与已定义的参数、集合或变量的名称相同,那么这些名称会被替换
为相应的值。这可能会导致错误,或者被解释为实际的值。
% Sets are referenced by the use of a \emph{template} tuple, consisting
% of placeholders which are replaced by the values of the components of
% the respective tuple. For example, a set $S$ consisting of two-dimensional
% tuples could be referenced by \code{<a,b> in S}. If any of the
% placeholders are actual values, only those tuples matching these
% values will be extracted.
% For example, \code{<1,b> in S} will only get
% those tuples whose first component is \code{1}. Please note that if
% one of the placeholders is the name of an already defined parameter,
% set or variable, it will be substituted. This will result either in an
% error or an actual value.

\paragraph{示例}
{\small
\begin{verbatim}
set A := { 1, 2, 3 };
set B := { "hi", "ha", "ho" };
set C := { <1,2,"x">, <6,5,"y">, <787,12.6,"oh"> };
\end{verbatim}
}
\noindent 对于集合表达式,表\ref{tab:zimpl-set-functions}中所示的函数和运
算符已被定义。
% \noindent For set expressions the functions and
% operators given in Table~\ref{tab:zimpl-set-functions} are defined.

关于形如 \emph{逻辑表达式} \code{then} \emph{集合表达式} \code{else} 
\emph{集合表达式} \code{end} 的语句形式如何使用的示例可以和下标集合的示例一同在
第\pageref{sec:indexed-sets}页找到。
% An example for the use of the \code{if} \emph{boolean-expression} \code{then}
% \emph{set-expression} \code{else} \emph{set-expression} \code{end} can
% be found on page~\pageref{sec:indexed-sets} together with the examples for indexed sets.

\paragraph{示例}
{\small
\begin{verbatim}
set D := A cross B;
set E := { 6 to 9 } union A without { 2, 3 };
set F := { 1 to 9 } * { 10 to 19 } * { "A", "B" };
set G := proj(F, <3,1>);
# 将会得到: { <"A",1>, <"A",2"> ... <"B",9> }
\end{verbatim}
% \begin{verbatim}
% set D := A cross B;
% set E := { 6 to 9 } union A without { 2, 3 };
% set F := { 1 to 9 } * { 10 to 19 } * { "A", "B" };
% set G := proj(F, <3,1>);
% # will give: { <"A",1>, <"A",2"> ... <"B",9> }
% \end{verbatim}
}

\begin{table}[htbp]
\centering
{\sffamily\small
\begin{tabular}{lp{50mm}p{70mm}} % For Chinese displaystyle.
% \begin{tabular}{llp{61mm}}
\toprule
\code{A*B},&\\
\code{A cross B}   &\raisebox{1ex}[0cm][0cm]{叉积}
                   &\raisebox{1ex}[0cm][0cm]{$\{(x,y)\mid x\in A\wedge y\in B\}$}\medskip\\
% \code{A*B},&\\
% \code{A cross B}   &\raisebox{1ex}[0cm][0cm]{cross product}
%                    &\raisebox{1ex}[0cm][0cm]{$\{(x,y)\mid x\in A\wedge y\in B\}$}\medskip\\
\code{A+B},&\\
\code{A union B}   &\raisebox{1ex}[0cm][0cm]{并集}
                   &\raisebox{1ex}[0cm][0cm]{$\{x\mid x\in A\vee x\in B\}$}\medskip\\
% \code{A+B},&\\
% \code{A union B}   &\raisebox{1ex}[0cm][0cm]{union}
%                    &\raisebox{1ex}[0cm][0cm]{$\{x\mid x\in A\vee x\in B\}$}\medskip\\
\code{union <i>}&\\
\code{ in I: S}&\raisebox{1ex}[0cm][0cm]{同上,对索引集合取并集}
               &\raisebox{1ex}[0cm][0cm]{$\bigcup_{i\in I}S_i$} \medskip\\
% \code{union <i>}&\\
% \code{ in I: S}&\raisebox{1ex}[0cm][0cm]{ditto, of indexed sets}
%                &\raisebox{1ex}[0cm][0cm]{$\bigcup_{i\in I}S_i$} \medskip\\
\code{A inter B}   &交集 & $\{x\mid x\in A\wedge x\in B\}$\medskip\\
% \code{A inter B}   &intersection & $\{x\mid x\in A\wedge x\in B\}$\medskip\\
\code{inter <i>}&\\
\code{ in I: S}&\raisebox{1ex}[0cm][0cm]{同上,对索引集合取交集}
               &\raisebox{1ex}[0cm][0cm]{$\bigcap_{i\in I}S_i$} \medskip\\
% \code{inter <i>}&\\
% \code{ in I: S}&\raisebox{1ex}[0cm][0cm]{ditto, of indexed sets}
%                &\raisebox{1ex}[0cm][0cm]{$\bigcap_{i\in I}S_i$} \medskip\\
\code{A$\setminus$B, A-B},&\\
\code{A without B} &\raisebox{1ex}[0cm][0cm]{差集}
                   &\raisebox{1ex}[0cm][0cm]{$\{x\mid x\in A\wedge x\not\in B\}$}\medskip\\
% \code{A$\setminus$B, A-B},&\\
% \code{A without B} &\raisebox{1ex}[0cm][0cm]{difference}
%                    &\raisebox{1ex}[0cm][0cm]{$\{x\mid x\in A\wedge x\not\in B\}$}\medskip\\
\code{A symdiff B} &对称差&
   $\{x\mid (x\in A\wedge x\not\in B)\vee(x\in B\wedge x\not\in A)\}$\\
% \code{A symdiff B} &symmetric difference&
%    $\{x\mid (x\in A\wedge x\not\in B)\vee(x\in B\wedge x\not\in A)\}$\\
\code{\{n\,{..}\,m \emph{by s}\}},& 生成集合, &
   $\{x\mid x=\min(n,m) + i|s| \leq \max(n,m),$\\
   &(默认$s = 1$) & $i\in\NN_0, x,n,m,s\in\ZZ\}$\\
% \code{\{n\,{..}\,m \emph{by s}\}},& generate, &
%    $\{x\mid x=\min(n,m) + i|s| \leq \max(n,m),$\\
%    &(default $s = 1$) & $i\in\NN_0, x,n,m,s\in\ZZ\}$\\
\code{\{n to m \emph{by s}\}}&生成集合 &
   $\{x\mid x=n + is \leq m, i\in\NN_0, x,n,m,s\in\ZZ\}$\\
% \code{\{n to m \emph{by s}\}}&generate &
%    $\{x\mid x=n + is \leq m, i\in\NN_0, x,n,m,s\in\ZZ\}$\\
\code{proj(A, t)}& 投影 &
   新的集合将由 $n$ 元组组成,其中第 $i$ 个 \\
   &$t=(e_1,\ldots,e_n)$&
   分量是集合 $A$ 中的第 $e_i$ 个分量。\\
% \code{proj(A, t)}& projection &
%    The new set will consist of $n$-tuples, with\\
%    &$t=(e_1,\ldots,e_n)$&
%    the $i$-th component being the $e_i$-th component of $A$.\\
\code{argmin <i>}&\\
\code{ in I : e(i)} & \raisebox{1ex}[0cm][0cm]{极小值}
  & \raisebox{1ex}[0cm][0cm]{$\argmin_{i\in I} e(i)$\medskip}\\
% \code{argmin <i>}&\\
% \code{ in I : e(i)} & \raisebox{1ex}[0cm][0cm]{minimum argument}
%    &$\argmin_{i\in I} e(i)$\medskip\\
\code{argmin(n) <i>} & 
   & 新集合将由满足使$e(i)$最小的$n$个元素 \\
\code{ in I : e(i)} & 取$n$个极小值
   & $i$组成。结果可能存在歧义 (由于不同元素可能取得相同的$e(i)$值)。\\
% \code{argmin(n) <i>} & $n$ minimum
%    & The new set consists\\
% \code{ in I : e(i)} &  arguments
%    &of those $n$ elements of $i$ for which $e(i)$ was
%      smallest. The result can be ambiguous.\\
\code{argmax <i>}&\\
\code{ in I : e(i)} & \raisebox{1ex}[0cm][0cm]{极大值}
   &$\argmax_{i\in I} e(i)$\medskip\\
% \code{argmax <i>}&\\
% \code{ in I : e(i)} & \raisebox{1ex}[0cm][0cm]{maximum argument}
%    &$\argmax_{i\in I} e(i)$\medskip\\
\code{argmin(n) <i>} & 
   & 新集合将由满足使$e(i)$最大的$n$个元素 \\
\code{ in I : e(i)} & 取$n$个极大值
   & $i$组成。结果可能存在歧义 (由于不同元素可能取得相同的$e(i)$值)。\\
% \code{argmax(n) <i>} & $n$ maximum
%    & The new set consists\\
% \code{ in I : e(i)} & arguments
%    & of those $n$ elements of $i$ for which $e(i)$ was
%      biggest. The result can be ambiguous.\\
\code{if a then b}\\
\code{else c end}        &\raisebox{1ex}[0cm][0cm]{条件判断}
   &\raisebox{1ex}[0cm][0cm]{$\left\{\begin{array}{rl}b,&\text{if }
   a=\text{true}\\c,&\text{if } a=\text{false}\end{array}\right.$}\\\\
% \code{if a then b}\\
% \code{else c end}        &\raisebox{1ex}[0cm][0cm]{conditional}
%    &\raisebox{1ex}[0cm][0cm]{$\left\{\begin{array}{rl}b,&\text{if }
%    a=\text{true}\\c,&\text{if } a=\text{false}\end{array}\right.$}\\\\
\code{permutate(A)} & 排列元素 & 生成一个包含集合 $A$ 中所有元素的排列的元组 \\  
% \code{permutate(A)} & generate a set consisting of tuples with all
%    permutations of the elements of $A$ &\\  
\bottomrule
\end{tabular}
}
\caption{集合关系函数}%
% \caption{Set related functions}
\label{tab:zimpl-set-functions}
\end{table}

\subsubsection{条件集合}
% \subsubsection{Conditional sets}
集合可以通过一个逻辑表达式表达式加以限定从而取得满足条件的元组。对于通过
\code{with}子句给定的表达式,会对集合中的每个元组进行求值。只有当表达式的
结果为\emph{true}时,相关元组才会被包含在新的集合中。
% It is possible to restrict a set to tuples that
% satisfy a Boolean expression. The expression given by the \code{with}
% clause is evaluated for each tuple in the set and only tuples for
% which the expression evaluates to \emph{true} are included in the new set.

\paragraph{示例}
{\small
\begin{verbatim}
set F := { <i,j> in Q with i > j and i < 5 };
set A := { "a", "b", "c" };
set B := { 1, 2, 3 };
set V := { <a,2> in A*B with a == "a" or a == "b" };
# 将会得到: { <"a",2>, <"b",2> }
set W := argmin(3) <i,j> in B*B : i+j;
# 将会得到: { <1,1>, <1,2>, <2,1> }
\end{verbatim}
% \begin{verbatim}
% set F := { <i,j> in Q with i > j and i < 5 };
% set A := { "a", "b", "c" };
% set B := { 1, 2, 3 };
% set V := { <a,2> in A*B with a == "a" or a == "b" };
% # will give: { <"a",2>, <"b",2> }
% set W := argmin(3) <i,j> in B*B : i+j;
% # will give: { <1,1>, <1,2>, <2,1> }
% \end{verbatim}
}

\subsubsection{索引集合}%
% \subsubsection{Indexed sets}
\label{sec:indexed-sets}
可以使用一个集合对另一个集合进行索引,从而得到一个“集合的集合”。
索引集合的访问方式是在集合名后加上方括号“\code{[}”和“\code{]}”,例如 \code{S[7]}。
表~\ref{tab:zimpl-idxset-fun} 列出了可用的函数。  
对索引集合的赋值有三种方式:  
\begin{itemize}
\setlength{\itemsep}{0pt}%
\item 赋值表达式可以是一个由逗号分隔的键值对列表,每对元素由索引集合中的一个元组和要赋值的集合表达式组成。  
\item 如果索引中包含一个索引元组,例如 \verb|<i> in I|,则赋值操作会对索引元组的每个取值分别求解。  
\item 通过返回索引集合的函数进行赋值。  
\end{itemize}
% It is possible to index one set with another set resulting in a set of sets.
% Indexed sets are accessed by adding the index of the set in brackets
% \code{[} and \code{]}, like \code{S[7]}.
% Table~\ref{tab:zimpl-idxset-fun} lists the available functions.
% There are three possibilities how to assign to an indexed set:
% \begin{itemize}
% \setlength{\itemsep}{0pt}%
% \item The assignment expression is a list of comma-separated pairs,
%       consisting of a tuple from the index set and a set expression to assign.
% \item If an index tuple is given as part of the index,
%       \eg \verb|<i> in I|, the assignment is evaluated for each value
%       of the index tuple.
% \item By use of a function that returns an indexed set.
% \end{itemize}

\subsubsection{示例}
{\small
\begin{verbatim}
set I           := { 1..3 };
set A[I]        := <1> {"a","b"}, <2> {"c","e"}, <3> {"f"};
set B[<i> in I] := { 3 * i };
set P[]         := powerset(I);
set J           := indexset(P);
set S[]         := subsets(I, 2);
set T[]         := subsets(I, 1, 2);
set K[<i> in I] := if i mod 2 == 0 then { i } else { -i } end;
set U           := union <i> in I : A[i];
set IN          := inter <j> in J : P[j]; # empty!
\end{verbatim}
}

\begin{table}[htbp]
\centering
{\sffamily\small
\begin{tabular}{lp{6.4cm}p{4cm}} % For Chinese displaystyle.
% \begin{tabular}{llp{5cm}}
\toprule
\code{powerset(A)} & 生成集合 $A$ 的所有子集 & $\{X\mid X\subseteq A\}$\\
\code{subsets(A,n)} & 生成集合 $A$ 包含 $n$ 个元素的子集
                    & $\{X\mid X\subseteq A\wedge |X|=n\}$\\
\code{subsets(A,n,m)}& 生成集合 $A$ 的包含 $n$ 到 $m$ 个元素的子集 
                    & $\{X\mid X\subseteq A\wedge n\leq |X|\leq m\}$\\
\code{indexset(A)}& 生成集合 $A$ 的索引集 & $\{1\ldots |A|\}$\\
\bottomrule
% \code{powerset(A)}& generates all subsets of $A$&$\{X\mid X\subseteq A\}$\\
% \code{subsets(A,n)}& generates all subsets of $A$\\
%                     & with $n$ elements&$\{X\mid X\subseteq A\wedge |X|=n\}$\\
% \code{subsets(A,n,m)}& generates all subsets of $A$\\
%                     & with between $n$ and $m$ elements
%                     &$\{X\mid X\subseteq A\wedge n\leq |X|\leq m\}$\\
% \code{indexset(A)}&the index set of $A$&$\{1\ldots |A|\}$\\
% \bottomrule
\end{tabular}
}
\caption{索引集函数}%
% \caption{Indexed set functions}
\label{tab:zimpl-idxset-fun}
\end{table}

%------------------------------------------------------------------------------
\subsection{参数}%
% \subsection{Parameters}
\label{ssec:parameters}
%------------------------------------------------------------------------------

参数(Parameter)是\zimpl 中定义常数的一种方式。参数既可以带有索引集合,
也可以不带索引。没有索引的参数只是一个单独的值,可以是数字或字符串;
带索引的参数则为索引集合中的每个元素指定一个对应的值。此外,还可以为参数
声明一个\emph{default}值。
% Parameters are the way to define constants within \zimpl. They can be
% declared with or without an index set. Without index a parameter is
% just a single value which is either a number or a string. For indexed
% parameters there is one value for each member of the set. It is
% possible to declare a \emph{default} value.

参数的声明方式如下所示:
使用关键字\code{param},紧跟参数名,并可以选择是否在方括号中给出索引集合。
接着在赋值符号之后,给出一个键值对列表。每对元素的第一个分量是来自索引集合
的元组,第二个分量则是该索引对应的参数值。如果只给出一个单独的值,
则该值会被赋给参数的所有索引成员。
% Parameters are declared in the following way:
% The keyword \code{param} is followed by the name of the parameter
% optionally followed by the index set in square brackets.
% Then, after the assignment operator, there is a list of pairs. The first element of each
% pair is a tuple from the index set, while the second element is the value of
% the parameter for this index. If there is only a single value given,
% it will be assigned to all members of the parameter.

\subsubsection{示例}
{\small
\begin{verbatim}
set A := { 12 .. 30 };
set C := { <1,2,"x">, <6,5,"y">, <3,7,"z"> };
param q := 5;
param r[C] := 7;         # 所有元素都是 7
param r[C] := default 7; # 和上一行一样
param u[A] := <13> 17, <17> 29, <23> 14 default 99;
param str[A] := <13> "hallo", <17> "tach" default "moin";
param amin := min A;               # = 12
param umin := min <a> in A : u[a]; # = 14
param mmax := max <i> in { 1 .. 10 } : i mod 5;
param w[C] := <1,2,"x"> 1/2, <6,5,"y"> 2/3;
param x[<i> in { 1 .. 8 } with i mod 2 == 0] := 3 * i;
\end{verbatim}
% \begin{verbatim}
% set A := { 12 .. 30 };
% set C := { <1,2,"x">, <6,5,"y">, <3,7,"z"> };
% param q := 5;
% param r[C] := 7;         # all members are 7
% param r[C] := default 7; # same as line above
% param u[A] := <13> 17, <17> 29, <23> 14 default 99;
% param str[A] := <13> "hallo", <17> "tach" default "moin";
% param amin := min A;               # = 12
% param umin := min <a> in A : u[a]; # = 14
% param mmax := max <i> in { 1 .. 10 } : i mod 5;
% param w[C] := <1,2,"x"> 1/2, <6,5,"y"> 2/3;
% param x[<i> in { 1 .. 8 } with i mod 2 == 0] := 3 * i;
% \end{verbatim}
}
\noindent
赋值并不需要是完整的。在这个例子当中,并没有给出参数“\code{w}”下的索引
\code{$<$3,7,"z"$>$}对应的数值。只要该索引值在模型中从未被引用,
这种做法就是允许的。
% \noindent
% Assignments do not need to be complete.  In the example, no value is
% given for index \code{$<$3,7,"z"$>$} of parameter \code{w}. This is
% correct as long as it is never referenced.

\subsubsection{参数表}
% \subsubsection{Parameter tables}
可以通过表格的方式初始化多维索引参数。这在处理二维参数时尤其有用。
数据应组织为表格结构,表格边界由“\code{|}”符号标示。随后需提供一行作为表头,
表头包含列索引;此外,每一行也必须对应一个行索引。列索引必须是一维的,
而行索引则可以是多维的。每个条目的完整索引由“行索引”与“列索引”拼接而成。
表中的各项以逗号分隔,并且允许使用任意合法表达式。如下面的第三个示例所示,
在表格之后还可以追加一个由条目组成的列表。
% It is possible to initialize multi-dimensional indexed parameters from
% tables. This is especially useful for two-dimensional parameters.
% The data is put in a table structure with \code{$|$} signs on each
% margin. Then a headline with column indices has to be added, and one index
% for each row of the table is needed. The column index has to be
% one-dimensional, but the row index can be
% multi-dimensional. The complete index for the entry is built by
% appending the column index to the row index.
% The entries are separated by commas. Any valid expression is
% allowed here. As can be seen in the third example below, it is
% possible to add a list of entries after the table.



\subsubsection{示例}
{\small
\begin{verbatim}
set I := { 1 .. 10 };
set J := { "a", "b", "c", "x", "y", "z" };

param h[I*J] :=   | "a", "c", "x", "z"   |
                |1|  12,  17, 99,     23 |
                |3|   4,   3,-17, 66*5.5 |
                |5| 2/3, -.4,  3, abs(-4)|
                |9|   1,   2,  0,      3 | default -99;

param g[I*I*I] :=     | 1, 2, 3 |
                  |1,3| 0, 0, 1 |
                  |2,1| 1, 0, 1 |;

param k[I*I] :=  |  7,  8,  9 |
               |4| 89, 67, 55 |
               |5| 12, 13, 14 |, <1,2> 17, <3,4> 99;
\end{verbatim}
}

\noindent 最后的这个示例等同于:
% \noindent The last example is equivalent to:
{\small
\begin{verbatim}
param k[I*I] := <4,7> 89, <4,8> 67, <4,9> 55, <5,7> 12,
                <5,8> 13, <5,9> 14, <1,2> 17, <3,4> 99;
\end{verbatim}
}

%------------------------------------------------------------------------------
\subsection{从文件读取集合与参数}\label{initfromfile}
% \subsection{Initializing sets and parameters from a file}%
% \label{initfromfile}
%------------------------------------------------------------------------------
可以从文件中读取集合或者参数。其语法是:
% It is possible to load the values for a set or a parameter from a
% file. The syntax is:

\smallskip
\code{read} \emph{文件名} \code{as} \emph{模板名}
[\code{skip} \emph{n}] [\code{use} \emph{n}]
[\code{match} \emph{s}] [\code{comment} \emph{s}]
% \code{read} \emph{filename} \code{as} \emph{template}
% [\code{skip} \emph{n}] [\code{use} \emph{n}]
% [\code{match} \emph{s}] [\code{comment} \emph{s}]

\smallskip
\noindent\emph{文件名} 是指要读取的文件名称。
\emph{模板名} 是一个用于生成元组的模板字符串的名称。
来自每一行输入会被拆分为若干个字段。字段的拆分遵循以下规则:
行首与行尾的空格与制表符会被去除;
每当遇到空格、制表符、逗号、分号或冒号时,都会开始一个新字段;
被双引号包围的文本不会被拆分,并且双引号会被自动移除;
若某个字段处发生拆分,则拆分点两侧的空格与制表符也会被移除;
若拆分是由逗号、分号或冒号引起的,那么每出现一次该符号就会新建一个字段。
% \noindent\emph{filename} is the name of the file to read.
% \emph{template} is a string with a template for the tuples to
% generate. Each input line from the file is split into fields. The
% splitting is done according to the following rules:
% All leading and trailing space and tab characters in the line are removed.
% Whenever a space, tab, comma, semicolon or colon is encountered
% a new field is started. Text that is enclosed in double quotes is not
% split and the quotes are always removed. When a field is split all space
% and tab characters around the splitting point are removed.
% If the split is
% due to a comma, semicolon or colon, each occurrence of these
% characters starts a new field.

\subsubsection{示例}
% \subsubsection{Examples}
如下的每一行输入都包含三个字段:%
% All these lines have three fields:
{\small
\begin{verbatim}
Hallo;12;3
Moin   7  2
"Hallo, Peter"; "Nice to meet you" 77
,,2
\end{verbatim}
}

\noindent 对于元组的每一个组成部分,都需要指定使用哪一个字段,
并在其后标明该字段的类型:若字段应被解释为数值,则写 \code{n};
若应作为字符串处理,则写 \code{s}。在模板之后,
可以添加一些可选修饰项,顺序不限:
% \noindent For each component of the tuple, the number of the field
% to use for the value is given, followed by either \code{n} if the
% field should be interpreted as a number or \code{s} for a string.
% After the template, some optional modifiers can be given. The order
% does not matter.
\code{match} \emph{s} 会正则表达式 \emph{s} 应用于读取的每一行文本。
只有当该表达式与该行匹配时,该行才会被进一步处理。所使用的正则语法为 
POSIX 扩展的正则表达式语法。
% \code{match} \emph{s} compares the regular expression \emph{s} against
% the line read from the file. Only if the expression matches the
% line, it is processed further. POSIX extended regular expression syntax
% is used.
\code{comment} \emph{s} 指定一个字符列表,这些字符用于标识文件中的注释。
若在一行中遇到任一注释字符,则该行在该字符处被截断。
% \code{comment} \emph{s} sets a list of characters that start
% comments in the file. Each line is ended when any of the comment
% characters is found.
\code{skip} \emph{n} 表示跳过文件的前 \emph{n} 行内容。
% \code{skip} \emph{n} instructs to skip the first
% \emph{n} lines of the file.
\code{use} \emph{n} 限制读取的行数不超过 \emph{n} 行。
在读取文件的过程中,空行、注释行以及未匹配的行都将被跳过,且不计入
\code{use} 和 \code{skip} 的计数范围内。
% \code{use} \emph{n} limits the number of
% lines to use to \emph{n}.
% When a file is read, empty lines, comment lines, and unmatched lines
% are skipped and not counted for the \code{use} and \code{skip} clauses.

\subsubsection{示例}
% \subsubsection{Examples}
{\small
\verb|set P := { read "nodes.txt" as "<1s>" };|

\smallskip
\noindent\verb|nodes.txt:|\\
\verb|   Hamburg            | $\rightarrow$ $<$"Hamburg"$>$\\
\verb|   München            | $\rightarrow$ $<$"München"$>$\\
\verb|   Berlin             | $\rightarrow$ $<$"Berlin"$>$\\

\smallskip
\noindent\verb|set Q := { read "blabla.txt" as "<1s,5n,2n>" skip 1 use 2 };|

\smallskip
\noindent\verb|blabla.txt:|\\
\verb|   Name;Nr;X;Y;No     | $\rightarrow$ 跳过\\
% \verb|   Name;Nr;X;Y;No     | $\rightarrow$ skip\\
\verb|   Hamburg;12;x;y;7   | $\rightarrow$ $<$"Hamburg",7,12$>$\\
\verb|   Bremen;4;x;y;5     | $\rightarrow$ $<$"Bremen",5,4$>$\\
\verb|   Berlin;2;x;y;8     | $\rightarrow$ skip\\

\smallskip
\noindent\verb|param cost[P] := read "cost.txt" as "<1s> 2n" comment "#";|

\smallskip
\noindent\verb|cost.txt:|\\
\verb|   # 名称 价格        | $\rightarrow$ 跳过\\
% \verb|   # Name Price       | $\rightarrow$ skip\\
\verb|   Hamburg 1000       | $\rightarrow$ $<$"Hamburg"$>$ 1000\\
\verb|   München 1200       | $\rightarrow$ $<$"München"$>$ 1200\\
\verb|   Berlin  1400       | $\rightarrow$ $<$"Berlin"$>$ 1400\\

\smallskip
\noindent\verb|param cost[Q] := read "haha.txt" as "<3s,1n,2n> 4s";|

\smallskip
\noindent\verb|haha.txt:|\\
\verb|      1:2:ab:con1     | $\rightarrow$ $<$"ab",1,2$>$ "con1"\\
\verb|      2:3:bc:con2     | $\rightarrow$ $<$"bc",2,3$>$ "con2"\\
\verb|      4:5:de:con3     | $\rightarrow$ $<$"de",4,5$>$ "con3"\\
}

\noindent 与表格式输入一样,可以在 read 语句之后添加由元组或参数项组成的列表。
% \noindent As with table format input, it is possible to add a list of tuples or
% parameter entries after a read statement.
\subsubsection{示例}
% \subsubsection{Examples}
{\small
\begin{verbatim}
set A := { read "test.txt" as "<2n>", <5>, <6> };
param winniepoh[X] :=
   read "values.txt" as "<1n,2n> 3n", <1,2> 17, <3,4> 29;
\end{verbatim}
}


也可以将一个单独的值读取到某个参数中。在这种情况下,文件应只包含一行内容,
或者应通过添加 \code{use 1} 参数来指示读取语句仅读取一行。
% It is also possible to read a single value into a parameter. In this
% case, either the file should contain only a single line, or the read
% statement should be instructed by means of a \code{use 1} parameter
% only to read a single line.

\subsubsection{示例}%
% \subsubsection{Examples}
{\small
\begin{verbatim}
# 读取第五行的第四个值
param n := read "huhu.dat" as "4n" skip 4 use 1;
\end{verbatim}
% \begin{verbatim}
% # Read the fourth value in the fifth line
% param n := read "huhu.dat" as "4n" skip 4 use 1;
% \end{verbatim}
}

如需读取一个文件中所有的值到元组中,对于字符串值,可以使用 \code{"<s+>"}
的模板形式,对于数值,可以使用 \code{"<n+>"} 的模板形式。需要注意的是,
目前只支持每行最多 65536 个值。
% If all values in a file should be read into a set, this is possible by
% use of the \code{"<s+>"} template for string values, and \code{"<n+>"}
% for numerical values. Note, that currently at most 65536 values are allowed in a
% single line.


\subsubsection{示例}%
% \subsubsection{Examples}
{\small
\begin{verbatim}
# 读取文件里的所有值到一个集合中
set X := { read "stream.txt" as "<n+>" };

stream.txt:
1 2 3 7 9 5 6 23
63 37 88
4
87 27

# X := { 1, 2, 3, 7, 9, 5, 6, 23, 63, 37, 88, 4, 87, 27 };
\end{verbatim}
% \begin{verbatim}
% # Read all values into a set
% set X := { read "stream.txt" as "<n+>" };
% 
% stream.txt:
% 1 2 3 7 9 5 6 23
% 63 37 88
% 4
% 87 27
% 
% # X := { 1, 2, 3, 7, 9, 5, 6, 23, 63, 37, 88, 4, 87, 27 };
% \end{verbatim}
}

%------------------------------------------------------------------------------
\subsection{\emph{sum} 表达式}
% \subsection{\emph{sum}-expressions }
%------------------------------------------------------------------------------
求和表达式规定为如下所示的形式:
% Sums are stated in the form:

\smallskip
\code{sum} \emph{索引} \code{do} \emph{语句}
% \code{sum} \emph{index} \code{do} \emph{term}

\smallskip
\noindent 尽管求和表达式可以多级嵌套,但是简单地合并索引可能会更方便。
\emph{<索引>} 的一般形式形如:
% \noindent It is possible to nest several sum instructions, but it is
% probably more convenient to simply merge the indices.
% The general form of \emph{index} is:

\smallskip
\emph{元组} \code{in} \emph{集合} \code{with} \emph{逻辑表达式}

\smallskip
\noindent 可以用一个冒号“\code{:}”来代替 \code{do},也可以用竖划线“\code{|}”
来代替 \code{with}。组成 \emph{元组} 的元素的个数必须与组成 \emph{集合} 的元素
个数相匹配,\code{with} 和 \emph{逻辑表达式} 的这部分是可选的。\emph{集合}
这部分可以使用任何表达式的形式表述的集合。示例将会在下一节中给出。
% \noindent It is allowed to write a colon \code{:} instead of \code{do} and a
% vertical bar \code{|} instead of \code{with}.
% The number of components in the \emph{tuple} and in the
% members of the \emph{set} must match. The \code{with} part of an \emph{index} is
% optional. The \emph{set} can be any expression giving a set. Examples
% are given in the next section.

%------------------------------------------------------------------------------
\subsection{\emph{forall} 表达式}
% \subsection{\emph{forall}-statements}
%------------------------------------------------------------------------------
遍历表达式的一般形式为:
% The general forms are:

\smallskip
\code{forall} \emph{索引} \code{do} \emph{语句}
% \code{forall} \emph{index} \code{do} \emph{term}

\smallskip
\noindent 可以嵌套多个遍历表达式\footnote{
  \textbf{译者注}: \LaTeX{}原文在此处有一行被注释掉的文本,未被显示在PDF中。
  其内容是:“需要注意的是,\emph{sum} 表达式本身也是这里的一个 \emph{语句},
  因此可以嵌套或连接它们。”原作者似乎本想提醒读者 \emph{forall} 表达式和前文
  的 \emph{sum} 表达式可以相互嵌套,但是不知道为什么将该内容隐藏了。
}。
\emph{index} 的一般形式和 \emph{sum} 表达式中是相同的。示例将在下一节中给出。
% \noindent It is possible to nest several forall instructions.
%It should be noted, that a \code{sum}-expression is a \emph{term}
%itself, so it is possible to nest or concatenate them.
% The general form of \emph{index} equals that of
% \emph{sum}-expressions.  Examples are given in the next section.

\subsubsection{警告!关于 \emph{forall} 和 \emph{sum} 语句中的索引变量作用域}
% \subsubsection{Caveat! Scope rules for index variables
% in \emph{forall} and \emph{sum}}
若索引变量已经在 \emph{forall} 和 \emph{sum} 语句外部定义过,
则会被替换为其当前已有值。这对于函数定义中的变量也同样成立,
如果它们在外部已经定义过,也会发生相同的替换行为。
% Index variables which are already defined outside the \emph{forall}
% and \emph{sum} statement are replaced by their current value. This
% also happens for variables inside functions definitions which are
% already defined outside.

\subsubsection{示例}
% \subsubsection{Examples}
{\small
\begin{verbatim}
k = 5; sum <i,k> in {1..10}*{1..50}: i; # == 55,因为 k 为定值
forall <k> in K: sum <k,b> in K*{1..3}: z[k,b]; # 等于 |K| 乘以三个元素之和
\end{verbatim}
% \begin{verbatim}
% k = 5; sum <i,k> in {1..10}*{1..50}: i; # == 55 as k is fixed
% forall <k> in K: sum <k,b> in K*{1..3}: z[k,b]; # |K| * sum of 3 elements
% \end{verbatim}
}

%------------------------------------------------------------------------------
\subsection{函数定义}
% \subsection{Function definitions}
%------------------------------------------------------------------------------

在 \zimpl 中可以定义函数。函数的返回值必须是数值、字符串、布尔值或集合之一。  
函数的参数仅能为数值或字符串,但在函数体内部可以访问所有已声明的集合与参数。
。
% It is possible to define functions within \zimpl. The value a
% function returns has to be either a number, a string, a boolean, or a set.
% The arguments of a function can only be numbers or strings, but within
% the function definition it is possible to access all otherwise
% declared sets and parameters.

函数定义必须以 \code{defnumb}、\code{defstrg}、\code{defbool} 或 \code{defset}
开头,具体使用哪一个取决于函数的返回值类型。其后为函数名及参数名列表,
用括号括起来。接着是赋值运算符 \code{:=},
后面需给出一个合法的表达式或集合表达式。
% The definition of a function has to start with
% \code{defnumb}, \code{defstrg}, \code{defbool}, or \code{defset}, 
% depending on the
% return value.
% Next is the name
% of the function and a list of argument names put in parentheses.
% An assignment operator \code{:=} has to follow and a valid
% expression or set expression.

\subsubsection{示例}
% \subsubsection{Examples}
{\small
\begin{verbatim}
defnumb dist(a,b)     := sqrt(a*a + b*b);
defstrg huehott(a)    := if a < 0 then "hue" else "hott" end;
defbool wirklich(a,b) := a < b and a >= 10 or b < 5;
defset  bigger(i)     := { <j> in K with j > i };
K = 5; defnum(a) := sum <i,k> in {1..3}*{1..5}: a; # == 3*a
\end{verbatim}
}


%------------------------------------------------------------------------------
\subsection{\emph{do print} 和 \emph{do check} 命令}
% \subsection{The \emph{do print} and \emph{do check} commands}
%------------------------------------------------------------------------------
\code{do} 命令比较特殊,它有两种可用的形式:\code{print} 和 \code{check}。
\code{print} 会将其后所跟的数值、字符串、布尔值、集合表达式或元组输出到标准输出
流。这通常用于检查集合是否包含预期的元素,或某些计算是否产生了预期的结果。
而\code{check} 则必须跟在一个布尔表达式前,若该表达式的值不为 \emph{true},
程序就会中止,并输出一条相应的错误信息。这可以用来断言某些条件是否成立。
可以在 \code{print} 或 \code{check} 语句之前使用 \code{forall} 子句。  
对于字符串或数值类型的值,也可以在 \code{print} 中给出逗号分隔的多个表达式。
% The \code{do} command is special.
% It has two possible incarnations:
% \code{print} and \code{check}. \code{print} will print to the
% standard output stream
% whatever numerical, string, Boolean or set expression, or tuple
% follows it. This can be used for example to check if a set has the
% expected members, or if some computation has the anticipated result.
% \code{check} always precedes a Boolean expression. If this
% expression does not evaluate to \emph{true}, the program is aborted
% with an appropriate error message. This can be used to assert that
% specific conditions are met.
% It is possible to use a \code{forall} clause before a \code{print}
% or \code{check} statement.
% For string and numeric values, it is possible to give a comma-separated
% list to the print statement.

\subsubsection{示例}
% \subsubsection{Examples}
{\small
\begin{verbatim}
set I := { 1..10 };
do print I;
do print "Cardinality of I:", card(I);
do forall <i> in I with i > 5 do print sqrt(i);
do forall <p> in P do check sum <p,i> in PI : 1 >= 1;
\end{verbatim}
}



%------------------------------------------------------------------------------
\section{模型}
% \section{Models}
%------------------------------------------------------------------------------
在本节中,我们将运用此前构建的各类机制来表述数学规划模型。  
除了用于声明变量、目标函数与约束的一般语法外,  
还提供了一些特殊的语法,用于简洁地表述特殊约束,  
例如特殊有序集 (special ordered sets) 、条件约束 
(conditional constraints) 以及扩展函数 (extended functions) 。
% In this section, we use the machinery developed up to now to
% formulate mathematical programs. Apart from the usual syntax to
% declare variables, objective functions and constraints there is
% special syntax available that permits the easy formulation of special
% constraints, such as special ordered sets, conditional constraints,
% and extended functions.

\subsection{规划变量}
% \subsection{Variables}
%------------------------------------------------------------------------------
与参数类似,变量也可以带有索引。
变量必须属于以下三种类型之一:实型 (称为\code{real})、
逻辑型 (\code{binary}) 或整型 (\code{integer})。默认类型为实型。
变量可以设置下界与上界。默认下界为零,上界为无穷大。
逻辑变量的取值范围始终在零与一之间。
可以根据变量的索引值来计算其上下界 (参见示例中的最后一条语句)。  
界限也可以显式设置为\code{infinity}或\code{-infinity}。  
逻辑与整数变量可以声明为\code{implicit},这也就是说,
即使变量被声明为连续型,在整数规划的最优解中也可以被假定为取整值。
\emph{implicit}的使用仅在配合支持该特性的求解器时才有意义。  
截至目前,仅有链接了\zimpl 的\scip (\url{http://scipopt.org}) 支持此特性。  
在其他情况下,该变量仅会被视为连续变量。
可使用\code{startval} 关键字为整数变量指定初始值。  
此外,可以使用\code{priority}关键字指定分支优先级。  
目前,当使用\code{-r}命令行选项时,这些值会被写入一个\cplex 格式的
\code{ord}分支顺序文件中。
% Like parameters, variables can be indexed.
% A variable has to be one out of three possible types:
% Continuous (called \code{real}), \code{binary} or \code{integer}. The default type is real.
% Variables may have lower and upper bounds. Defaults are
% zero as the lower and infinity as the upper bound. Binary variables are
% always bounded between zero and one.
% It is possible to compute the value of the lower or upper bounds
% depending on the index of the variable (see the last declaration in the
% example). Bounds can also be set to \code{infinity} and \code{-infinity}.
% Binary and integer variables can be declared \code{implicit}, \ie the
% variable can be assumed to have an integral value in any optimal
% solution to the integer program, even if it is declared continuous.
% Use of \emph{implicit} is only useful if used together with a solver
% that take advantage of this information. As of this writing, only \scip
% (\url{http://scipopt.org}) with linked in \zimpl can do so.
% In all other cases, the variable is treated as a continuous variable.
% It is possible to specify initial values for integer variables by use
% of the \code{startval} keyword. Furthermore, the branching  priority
% can be given using the \code{priority} keyword. Currently, these
% values are written to a \cplex \code{ord} branching order file if the
% \code{-r} command line switch is given.

\subsubsection{示例}
% \subsubsection{Examples}
{\small
\begin{verbatim}
var x1;
var x2 binary;
var x3 integer >= -infinity      # 自由变量
var y[A] real >= 2 <= 18;
var z[<a,b> in C] integer
      >= a * 10 <= if b <= 3 then p[b] else infinity end;
var w implicit binary;
var t[k in K] integer >= 1 <= 3 * k startval 2 * k priority 50;
\end{verbatim}
% \begin{verbatim}
% var x1;
% var x2 binary;
% var x3 integer >= -infinity      # free variable
% var y[A] real >= 2 <= 18;
% var z[<a,b> in C] integer
%       >= a * 10 <= if b <= 3 then p[b] else infinity end;
% var w implicit binary;
% var t[k in K] integer >= 1 <= 3 * k startval 2 * k priority 50;
% \end{verbatim}
}

%\fbox{Remember: if nothing is specified a lower bound of zero is assumed.}


\subsection{目标函数}
% \subsection{Objective}
在模型的声明中,最多只能声明一个目标函数。目标函数可以是\code{maximize}
(极大化) 或者\code{minimize} (极小化) 的。紧随关键字声明值后的是一个 
(目标函数的) 名字,以及一个冒号“\code{:}”随后是一个目标函数形式的一个
线性表达式。
% There must be at most one objective statement in a model. The objective
% can be either \code{minimize} or \code{maximize}. Following the
% keyword is a name, a colon : and then a linear term expressing the
% objective function.

如果规划目标中包含补偿值 (offset),也就是说一个常数值,\zimpl 会自动生成一个
内部变量\code{@@ObjOffset}并将其赋值为1。该变量将以合适的系数被纳入目标函数中。
\footnote{
  这是因为,无论是 \lpf 还是 \mps 格式,都没有通用的方法在目标函数中直接表示
  一个常数项。
  \textbf{译者注}: 我不知道原文作者此处所说的“offset”是什么意思,姑且将其翻译
  为“补偿值”。除此之外,笔者也并未发现\lpf 文件不支持目标函数里包含常数项的语
  法形式。这可能是为了兼容一些比较旧的求解器机制而作的设计。
}
% If there is an objective offset, \ie a constant value in the
% objective function, \zimpl automatically generates an internal
% variable \code{@@ObjOffset} set to one. This variable is put into the
% objective function with the appropriate coefficient.\footnote{The reason for
% this is that there is no portable way to put an offset into the
% objective function in neither \lpf nor \mps-format.}

\subsubsection{Example}
{\small
\begin{verbatim}
minimize cost: 12 * x1 -4.4 * x2 + 5
   + sum <a> in A : u[a] * y[a]
   + sum <a,b,c> in C with a in E and b > 3 : -a/2 * z[a,b,c];
maximize profit: sum <i> in I : c[i] * x[i];
\end{verbatim}
}

%------------------------------------------------------------------------------
\subsection{Constraints}\label{ssec:ug:constraints}
%------------------------------------------------------------------------------
The general format for a constraint is:

\smallskip
\verb|subto name: term sense term|

\smallskip
\noindent Alternatively, it is also possible to define \emph{ranged} constraints
which have the form:

\smallskip
\verb|subto name: expr sense term sense expr|

\smallskip
\noindent \code{name} can be any name starting with a letter. \code{term} is defined
as in the objective. \code{sense} is one of
\code{<=}, \code{>=} and \code{==}. In case of ranged constraints
both senses have to be equal and may not be \code{==}. \code{expr} is any
valid expression that evaluates to a number.

Additional to linear constraints as in the objective it is also
possible to state terms with higher degree for constraints.

Many constraints can be generated with one statement by the use of the
\code{forall} instruction, as shown below.

\subsubsection{Examples}
{\small
\begin{verbatim}
subto time:  3 * x1 + 4 * x2 <= 7;
subto space: 50 >= sum <a> in A: 2 * u[a] * y[a] >= 5;
subto weird: forall <a> in A: sum <a,b,c> in C: z[a,b,c]==55;
subto c21: 6*(sum <i> in A: x[i] + sum <j> in B : y[j]) >= 2;
subto c40: x[1] == a[1] + 2 * sum <i> in A do 2*a[i]*x[i]*3+4;
subto frown: forall <i,j> in X cross { 1 to 5 } without { <2,3> }
               with i > 5 and j < 2 do
                  sum <i,j,k> in X cross { 1 to 3 } cross Z do
                     p[i] * q[j] * w[j,k] >= if i == 2 then 17 else 53 end;
subto nonlin: 3 * x * y * z + 6 <= x^3 + z * y^2 + 3;
\end{verbatim}
}
\noindent Note that in the example \emph{i} and \emph{j} are set by the \code{forall}
instruction. So they are fixed in all invocations of \code{sum}.

%------------------------------------------------------------------------------
\subsubsection{\emph{if} in terms}
%------------------------------------------------------------------------------
Part of a constraint can be put into an \code{if-then-else-end}.
In this particular case, the else part is mandatory and both parts
need to include some variables in the term.

\subsubsection{Examples}
{\small
\begin{verbatim}
subto c2: sum <i> in I :
     if (i mod 2 == 0) then 3 * x[i] else -2 * y[i] end <= 3;
\end{verbatim}
}

%------------------------------------------------------------------------------
\subsubsection{\emph{if} in constraints}
%------------------------------------------------------------------------------
It is possible to put two variants of a constraint into an
\code{if}-statement.
A \code{forall} statement inside the result part of an \code{if} is
also possible. The \code{else} part is optional.

\subsubsection{Examples}
{\small
\begin{verbatim}
subto c1: forall <i> in I do
  if (i mod 2 == 0) then  3 * x[i] >= 4
                    else -2 * y[i] <= 3 end;
\end{verbatim}
}

%------------------------------------------------------------------------------
\subsubsection{Combining constraints with \emph{and}}
%------------------------------------------------------------------------------
It is possible to group constraints by concatenating them with
\code{and}. This group will then always be processed together.
This operator is particularly useful in combination with \code{forall} and \code{if}.

\subsubsection{Examples}
{\small
\begin{verbatim}
subto c1: forall <i> in I:
  if (i > 3)
     then if (i == 4)
        then z[1] + z[2] == i
        else z[2] + z[3] == i
     end and
     z[3] + z[4] == i and
     z[4] + z[5] == i
  else
     if (i == 2)
        then z[1] + z[2] == i + 10
        else z[2] + z[3] == i + 10
     end and
     z[3] + z[4] == i + 10 and
     z[4] + z[5] == i + 10
  end;
\end{verbatim}
}


%------------------------------------------------------------------------------
\subsubsection{Constraint attributes}
%------------------------------------------------------------------------------
It is possible to specify special attributes for a constraint regarding
how it should be handled later on.

\begin{description}
\item[scale]
  Before the constraint is written to a file, it is scaled by
  $1/\max |c|$ with $c$ being the coefficients of the constraint.
\item[separate]
  Do not include the constraint in the initial LP, but separate it later on.
  The constraint need not to be checked for feasibility.
  When written to an \lpf file it will be written into a \code{USER
  CUTS} section, in \scip separate will be set to true.
\item[checkonly]
  Do not include the constraint in the initial LP, but separate only to
  check for feasibility.
  When written to an \lpf file it will be written into a \code{LAZY
  CUTS} section, in \scip check will be set to true.
\item[indicator]
  If vif (see below) is part of the constraint then it is modeled as an
  indicator constraint and not by a big-M formulation.
  If you use this attribute, you can no longer write \mps-format, as
  indicator variables are not supported by this format.
\item[qubo]
  Transform the constraint into an qudratic objective.
  Note that neccessary binary slacks are created automatically.
  However, integer variables are currently not automatically
  converted into binary variables.
\item[penaltyX] Only to be used together with the \emph{qubo}
attribute. Multiply the constraint that is put into the objective
by a factor of $P=10^X$, $X\in\{1,\ldots,6\}$.
\end{description}

\noindent
The attributes are given after the constraint, before the semi-colon
and are separated by commas.

\subsubsection{Examples}
{\small
\begin{verbatim}
subto c1: 1000 * x + 0.3 * y <= 5, scale, checkonly;
subto c2: x + y + z == 7, separate;
subto c3: vif x == 1 then y == 7 end, indicator;
subto c4: sum <i> in I : x[i] <= 15, qubo, penalty3;
\end{verbatim}
}


%------------------------------------------------------------------------------
\subsubsection{Special ordered sets}
%------------------------------------------------------------------------------
\zimpl can be used to specify special ordered sets (\sos) for an
integer program. If a model contains any \sos a \code{sos} file is
written together with the \code{lp} or \code{mps} file.
The general format of a special ordered set is:

\smallskip
\verb@sos name: [type1|type2] priority expr : term@

\smallskip
\noindent \code{name} can be any name starting with a letter. \sos use
the same namespace as constraints.
\code{term} is defined as in the objective. \code{type1} or
\code{type2} indicate whether a type-1 or type-2 special ordered set
is declared. The priority is optional and equal to the priority
setting for variables.
Many \sos can be generated with one statement by the use of the
\code{forall} instruction, as shown above.

\subsubsection{Examples}
{\small
\begin{verbatim}
sos s1: type1: 100 * x[1] + 200 * x[2] + 400 * x[3];
sos s2: type2 priority 100 : sum <i> in I: a[i] * x[i];
sos s3: forall <i> in I with i > 2:
           type1: (100 + i) * x[i] + i * x[i-1];
\end{verbatim}
}

%------------------------------------------------------------------------------
\subsubsection{Extended constraints}
%------------------------------------------------------------------------------
\zimpl can generate systems of constraints that
mimic conditional constraints. The general syntax is as follows (note
that the \code{else} part is optional):

\smallskip
\centerline{
\code{vif} \emph{boolean-constraint} \code{then} \emph{constraint}
[ \code{else} \emph{constraint} ] \code{end}}

\smallskip
\noindent where \emph{boolean-constraint} consists of a linear expression involving
variables. All these variables have to be bounded integer or binary
variables. It is not possible to use any continuous variables or integer
variables with infinite bounds in a \emph{boolean-constraint}.
All comparison operators ($<$, $\le$, $==$, $!\!\!=$, $\ge$, $>$) are
allowed. Also, the combination of several terms with \code{and},
\code{or}, and \code{xor} and negation with \code{not} is possible.
The conditional constraints (those which follow after \code{then} or
\code{else}) may include bounded continuous variables.
Be aware that using this construct will lead to
the generation of several additional constraints and variables.

\subsubsection{Examples}
{\small
\begin{verbatim}
var x[I] integer >= 0 <= 20;
subto c1: vif 3 * x[1] + x[2] != 7
   then sum <i> in I : y[i] <= 17
   else sum <k> in K : z[k] >= 5 end;
subto c2: vif x[1] == 1 and x[2] > 5 then x[3] == 7 end;
subto c3: forall <i> in I with i < max(I) :
   vif x[i] >= 2 then x[i + 1] <= 4 end;
\end{verbatim}
}

%------------------------------------------------------------------------------
\subsubsection{Extended functions}
%------------------------------------------------------------------------------
It is possible to use special functions on terms with variables that
will automatically be converted into a system of inequalities. The
arguments of these functions have to be linear terms consisting of
bounded integer or binary variables. At the moment only the function
\code{vabs(t)} that computes the absolute value of the term \code{t}
is implemented, but functions like the minimum or the maximum of two
terms, or the sign of a term can be implemented in a similar manner.
Again, using this construct will lead to
the generation of several additional constraints and variables.

\subsubsection{Examples}
{\small
\begin{verbatim}
var x[I] integer >= -5 <= 5;
subto c1: vabs(sum <i> in I : x[i]) <= 15;
subto c2: vif vabs(x[1] + x[2]) > 2 then x[3] == 2 end;
\end{verbatim}
}




\clearpage
\section{Modeling examples}\label{modelingexamples}
In this section we show some examples of well-known problems
translated into \zimpl format.

\subsection{The diet problem}
This is the first example in
\cite[Chapter 1, page 3]{Chvatal1983}.
It is a classic so-called \emph{diet} problem, see for example
\cite{Dantzig1990} about its implications in practice.

Given a set of foods $F$ and a set of nutrients $N$, we have a table
$\pi_{fn}$ of the amount of nutrient $n$ in food $f$. Now $\Pi_n$
defines how much intake of each nutrient is needed. $\Delta_f$
denotes for each food the maximum number of servings acceptable.
Given prices $c_f$ for each food, we have to find a selection of foods
that obeys the restrictions and has minimal cost. An integer
variable $x_f$ is introduced for each $f\in F$ indicating the number of servings of
food $f$. Integer variables are used, because only complete servings can be
obtained, \ie half an egg is not an option.
The problem may be stated as:
\begin{align*}
\min&\sum_{f\in F}c_{f} x_{f}&&\mbox{subject to}\nonumber\\
\sum_{f\in F} \pi_{fn} x_{f}&\ge\Pi_n&&\fa n\in N\nonumber\\
0\leq x_{f}&\leq\Delta_f&&\fa f\in F\nonumber\\
x_{f}&\in\NNZ&&\fa f\in F
\end{align*}

\noindent This translates into \zimpl as follows:

\medskip
\lstset{language=zimpl,%
basicstyle=\sffamily\footnotesize,%
numberstyle=\sffamily\tiny\color{siennabrown},stepnumber=1}
\begin{lstlisting}[frame=tb]{}
set Food      := { "Oatmeal", "Chicken", "Eggs",
                   "Milk",    "Pie",     "Pork" };
set Nutrients := { "Energy",  "Protein", "Calcium" };
set Attr      := Nutrients + { "Servings", "Price" };

param needed[Nutrients] :=
   <"Energy"> 2000, <"Protein"> 55, <"Calcium"> 800;

param data[Food * Attr] :=
          |"Servings","Energy","Protein","Calcium","Price"|
|"Oatmeal"|        4 ,    110 ,       4 ,       2 ,     3 |
|"Chicken"|        3 ,    205 ,      32 ,      12 ,    24 |
|"Eggs"   |        2 ,    160 ,      13 ,      54 ,    13 |
|"Milk"   |        8 ,    160 ,       8 ,     284 ,     9 |
|"Pie"    |        2 ,    420 ,       4 ,      22 ,    20 |
|"Pork"   |        2 ,    260 ,      14 ,      80 ,    19 |;
#                       (kcal)       (g)      (mg) (cents)

var x[<f> in Food] integer >= 0 <= data[f, "Servings"];

minimize cost: sum <f> in Food : data[f, "Price"] * x[f];

subto need: forall <n> in Nutrients do
   sum <f> in Food : data[f, n] * x[f] >= needed[n];
\end{lstlisting}

\medskip
\noindent The cheapest meal satisfying all requirements costs 97 cents and
consists of four servings of oatmeal, five servings of milk and two
servings of pie.

%x$Oatmeal                     4.000000
%x$Milk                        5.000000
%x$Pie                         2.000000

\subsection{The traveling salesman problem}
In this example, we show how to generate an exponential
description of the \emph{symmetric traveling salesman problem} (\tsp)
as given for example in
\cite[Section 58.5]{Schrijver2003}.

Let $G=(V,E)$ be a complete graph, with $V$ being the set of cities
and $E$ being the set of links between the cities. Introducing binary
variables $x_{ij}$ for each $(i,j)\in E$ indicating if edge $(i,j)$ is
part of the tour, the \tsp can be written as:
\begin{align*}
\min &\sum_{(i,j)\in E}d_{ij} x_{ij}&&\text{subject to}\nonumber\\
\sum_{(i,j)\in\delta_v} x_{ij}&=2&&\fa v\in V\nonumber\\
\sum_{(i,j)\in E(U)} x_{ij}&\leq |U|-1&&\fa U\subseteq V, \emptyset\neq
U\neq V\\
x_{ij}&\in\{0,1\}&&\fa (i,j)\in E\nonumber
\end{align*}
%
The data is read in from a file that gives the number of the city and the
x and y coordinates. Distances between cities are assumed Euclidean. For example:
{\footnotesize
\setlength\columnseprule{0.4pt}
\begin{multicols}{2}
\begin{verbatim}
# City       X    Y
Berlin     5251 1340
Frankfurt  5011  864
Leipzig    5133 1237
Heidelberg 4941  867
Karlsruhe  4901  840
Hamburg    5356  998
Bayreuth   4993 1159
Trier      4974  668
Hannover   5237  972
Stuttgart  4874  909
Passau     4856 1344
Augsburg   4833 1089
Koblenz    5033  759
Dortmund   5148  741
Bochum     5145  728
Duisburg   5142  679
Wuppertal  5124  715
Essen      5145  701
Jena       5093 1158
\end{verbatim}
\end{multicols}
}

% {\small
% \begin{verbatim}
% #City        x y
% "Sylt"       1 1
% "Flensburg"  3 1
% "Neumünster" 2 2
% "Husum"      1 3
% "Schleswig"  3 3
% "Ausacker"   2 4
% \end{verbatim}
% }

\noindent The formulation in \zimpl follows below. Please note that \code{P[]}
holds all subsets of the cities. As a result
19 cities is about as far as one can get with this approach.
Information on how to solve much larger instances can be found on the
\textsc{concorde} website\footnote{http://www.tsp.gatech.edu}.

\medskip
\lstset{language=zimpl,%
basicstyle=\sffamily\footnotesize,%
numberstyle=\sffamily\tiny\color{siennabrown},stepnumber=1}
\begin{lstlisting}[frame=tb]{}
set V             := { read "tsp.dat" as "<1s>" comment "#" };
set E             := { <i,j> in V * V with i < j };
set P[]           := powerset(V);
set K             := indexset(P);

param px[V]       := read "tsp.dat" as "<1s> 2n" comment "#";
param py[V]       := read "tsp.dat" as "<1s> 3n" comment "#";

defnumb dist(a,b) := sqrt((px[a]-px[b])^2 + (py[a]-py[b])^2);

var x[E] binary;

minimize cost: sum <i,j> in E : dist(i,j) * x[i, j];

subto two_connected: forall <v> in V do
   (sum <v,j> in E : x[v,j]) + (sum <i,v> in E : x[i,v]) == 2;

subto no_subtour:
   forall <k> in K with
      card(P[k]) > 2 and card(P[k]) < card(V) - 2 do
      sum <i,j> in E with <i> in P[k] and <j> in P[k] : x[i,j]
      <= card(P[k]) - 1;
\end{lstlisting}

\medskip
\noindent The resulting \lp has 171 variables, 239,925 constraints, and
22,387,149 non-zero entries in the constraint matrix, giving an \mps-file
size of 936\,\textsc{mb}. \cplex solves this to optimality
without branching in less than a minute.\footnote{Only 40
simplex iterations are needed to reach the optimal solution.}

An optimal tour for the data above is
Berlin, Hamburg, Hannover, Dortmund, Bo\-chum, Wuppertal, Essen,
Duisburg, Trier, Koblenz, Frankfurt, Heidelberg, Karlsruhe, Stuttgart,
Augsburg, Passau, Bayreuth, Jena, Leipzig, Berlin.


\subsection{The capacitated facility location problem}
Here we give a formulation of the \emph{capacitated facility
location} problem. It may also be considered as a kind of \emph{bin packing} problem
with packing costs and variable sized bins, or as a \emph{cutting stock} problem
with cutting costs.

Given a set of possible plants $P$ to build, and a set of stores $S$
with a certain demand $\delta_s$ that has to be satisfied, we have
to decide which plant should serve which store.
We have costs $c_p$ for building plant $p$ and $c_{ps}$
for transporting the goods from plant $p$ to store $s$.
Each plant has only a limited capacity $\kappa_p$.
We insist that each store is served by exactly one plant.
Of course we are looking for the cheapest solution:
%
\begin{align}
\min\sum_{p\in P}c_p z_p&
   +\!\!\!\sum_{p\in P, s\in S}\!\!\!c_{ps} x_{ps}&&\mbox{subject to}\nonumber\\
\sum_{p\in P} x_{ps}& = 1&&\fa s\in S\label{eqn:assign}\\
x_{ps}&\leq z_{p}&&\fa s\in S, p\in P\label{eqn:open}\\
\sum_{s\in S} \delta_s x_{ps}&\leq\kappa_p&&\fa p\in P\label{eqn:capacity}\\
x_{ps},z_p&\in\BB&&\fa p\in P, s\in S\nonumber
\end{align}
%
We use binary variables $z_p$ which are set to one, if and only if plant $p$ is
to be built. Additionally, we have binary variables $x_{ps}$
which are set to one if and only if plant $p$ serves shop $s$.
Equation~\eqref{eqn:assign} demands that each store is assigned to
exactly one plant. Inequality~\eqref{eqn:open} makes sure that a plant
that serves a shop is built. Inequality~\eqref{eqn:capacity}
assures that the shops are served by a plant which does not exceed its
capacity. Putting this into \zimpl yields the program shown on the
next page.
The optimal solution for the instance described by the program is to build plants \code{A} and
\code{C}. Stores 2, 3, and 4 are served by plant \code{A} and the others by
plant \code{C}. The total cost is 1457.

\clearpage
\lstset{language=zimpl,%
basicstyle=\sffamily\footnotesize,%
numberstyle=\sffamily\tiny\color{siennabrown},stepnumber=1}
\begin{lstlisting}[frame=]{}
set PLANTS := { "A", "B", "C", "D" };
set STORES := { 1 .. 9 };
set PS     := PLANTS * STORES;

# How much does it cost to build a plant and what capacity
# will it then have?
param building[PLANTS]:= <"A"> 500, <"B"> 600, <"C"> 700, <"D"> 800;
param capacity[PLANTS]:= <"A">  40, <"B">  55, <"C">  73, <"D">  90;

# The demand for each store
param demand  [STORES]:= <1> 10, <2> 14,
                         <3> 17, <4>  8,
                         <5>  9, <6> 12,
                         <7> 11, <8> 15,
                         <9> 16;

# Transportation cost from each plant to each store
param transport[PS] :=
      |  1,  2,  3,  4,  5,  6,  7,  8,  9 |
  |"A"| 55,  4, 17, 33, 47, 98, 19, 10,  6 |
  |"B"| 42, 12,  4, 23, 16, 78, 47,  9, 82 |
  |"C"| 17, 34, 65, 25,  7, 67, 45, 13, 54 |
  |"D"| 60,  8, 79, 24, 28, 19, 62, 18, 45 |;

var x[PS]     binary;  # Is plant p supplying store s ?
var z[PLANTS] binary;  # Is plant p built ?

# We want it cheap
minimize cost: sum <p> in PLANTS : building[p] * z[p]
             + sum <p,s> in PS : transport[p,s] * x[p,s];

# Each store is supplied by exactly one plant
subto assign:
  forall <s> in STORES do
     sum <p> in PLANTS : x[p,s] == 1;

# To be able to supply a store, a plant must be built
subto build:
   forall <p,s> in PS do
      x[p,s] <= z[p];

# The plant must be able to meet the demands from all stores
# that are assigned to it
subto limit:
   forall <p> in PLANTS do
      sum <s> in S : demand[s] * x[p,s] <= capacity[p];
\end{lstlisting}
\clearpage

\subsection{The \emph{n}-queens problem}
\label{ssec:example:n-queens-problem}
The problem is to place $n$ queens on a $n\times n$ chessboard so that no
two queens are on the same row, column or diagonal.
The $n$-queens problem is a classic combinatorial search problem
often used to test the performance of algorithms that solve satisfiability
problems. Note though, that there are algorithms available which need
linear time in practice, like, for example, those of \cite{SosicGu1991}.
We will show four different models for the problem and compare
their performance.

\subsubsection{The integer model}
\label{nqueens}
The first formulation uses
one general integer variable for each row of the board.
Each variable can assume the value of a column, \ie we have $n$ variables
with bounds $1\ldots n$. Next, we use the
\code{vabs} extended function to model an \emph{all different}
constraint on the variables (see constraint c1).
This makes sure that no queen is located
in the same column than any other queen.
The second constraint (c2) is used to block all the diagonals of a
queen by demanding that the absolute value of the row
distance and the column distance of each pair of queens are
different. We model $a\neq b$ by $\mbox{abs}(a-b)\geq 1$.

Note that this formulation only works if a queen can be placed in each
row, \ie if the size of the board is at least $4\times4$.

\medskip
%\noindent\textsf{Integer formulation}
\lstset{language=zimpl,%
basicstyle=\sffamily\footnotesize,%
numberstyle=\sffamily\tiny\color{siennabrown},stepnumber=1}
\begin{lstlisting}[frame=tb]{}
param queens := 8;

set C := { 1 .. queens };
set P := { <i,j> in C * C with i < j };

var x[C] integer >= 1 <= queens;

subto c1: forall <i,j> in P do vabs(x[i] - x[j]) >= 1;
subto c2: forall <i,j> in P do
             vabs(vabs(x[i] - x[j]) - abs(i - j)) >= 1;
\end{lstlisting}

\medskip
\noindent The following table shows the performance of the model. Since the
problem is modeled as a
pure satisfiability problem, the solution time depends only on how
long it takes to find a feasible solution.\footnote{Which is, in fact,
rather random.}
The columns titled \emph{Vars}, \emph{Cons}, and \emph{NZ} denote the
number of variables, constraints and non-zero entries in the
constraint matrix of the generated integer program.
\emph{Nodes} lists the number of branch-and-bound nodes evaluated by
the solver, and \emph{time} gives the solution time in \cpu seconds.
\begin{center}
{\sffamily\small
\begin{tabular}{crrrrrr}
\toprule
Queens & Vars & Cons &   NZ      & Nodes & Time [s]\\
\midrule
   8    &   344 &   392 &    951 &   1,324   & $<$1\\
  12    &   804 &   924 &  2,243 & 122,394   &  120\\
  16    & 1,456 & 1,680 &  4,079 & $>$1 mill.& $>$1,700\\
\bottomrule
\end{tabular}
}
\end{center}
As we can see, between 12 and 16 queens is the maximum instance size
we can expect to solve with this model. Neither changing the \cplex parameters
to aggressive cut generation nor setting emphasis on integer
feasibility improves the performance significantly.

\subsubsection{The binary models}\label{nqueens:binary}
Another approach to model the problem is to have
one binary variable for each square of the
board. The variable is one if and only if a queen is in this square
and we maximize the number of queens on the board.

For each square we compute in advance which other squares are blocked if a queen is
placed on this particular square. Then the extended \code{vif}
constraint is used to set the variables of the blocked squares to zero if a
queen is placed.

\medskip
%\noindent\textsf{Binary formulation A}
\lstset{language=zimpl,%
basicstyle=\sffamily\footnotesize,escapechar=@,%
numberstyle=\sffamily\tiny\color{siennabrown},stepnumber=1}
\begin{lstlisting}[frame=tb]{}
param columns := 8;

set C   := { 1 .. columns };
set CxC := C * C;

set TABU[<i,j> in CxC] := { <m,n> in CxC with (m != i or n != j)
   and (m == i or n == j or abs(m - i) == abs(n - j)) };

var x[CxC] binary;

maximize queens: sum <i,j> in CxC : x[i,j];

subto c1: forall <i,j> in CxC do vif x[i,j] == 1 then @\label{queens-vif}@
              sum <m,n> in TABU[i,j] : x[m,n] <= 0 end;
\end{lstlisting}

\medskip
\noindent Using extended formulations can make the models more comprehensible.
For example, replacing constraint c1 in line \ref{queens-vif} with
an equivalent one that does not use \code{vif} as shown below,
leads to a formulation that is much harder to understand.

\medskip
\lstset{language=zimpl,%
basicstyle=\sffamily\footnotesize,%
numberstyle=\sffamily\tiny\color{siennabrown},stepnumber=1}
\begin{lstlisting}[firstnumber=13]{}
subto c2: forall <i,j> in CxC do
             card(TABU[i,j]) * x[i,j]
           + sum <m,n> in TABU[i,j] : x[m,n] <= card(TABU[i,j]);
\end{lstlisting}

\medskip
\noindent After the application of the \cplex presolve procedure both
formulations result in identical integer programs. The performance of
the model is shown in the following table. \emph{S} indicates the
\cplex settings used: Either \emph{(D)efault},
\emph{(C)uts}\footnote{Cuts: \code{mip cuts all 2} and \code{mip strategy probing 3}.}, or
\emph{(F)easibility}\footnote{Feasibility: \code{mip cuts all -1} and \code{mip emph 1}}.
\emph{Root Node} indicates
the objective function value of the \lp relaxation of the root node.
\begin{center}
{\sffamily\small
\begin{tabular}{ccrrrrrrr}
\toprule
Queens & S & Vars & Cons &   NZ   & Root Node & Nodes & Time [s]\\
\midrule
   8   & D &   384 &   448 &   2,352 & 13.4301 &     241 & $<$1\\
       & C &       &       &         &  8.0000 &       0 & $<$1\\
  12   & D &   864 & 1,008 &   7,208 & 23.4463 &  20,911 & 4\\
       & C &       &       &         & 12.0000 &       0 & $<$1\\
  16   & D & 1,536 & 1,792 &  16,224 & 35.1807 & 281,030 & 1,662\\
       & C &       &       &         & 16.0000 &      54 & 8\\
  24   & C & 3,456 & 4,032 &  51,856 & 24.0000 &      38 & 42\\
  32   & C & 6,144 & 7,168 & 119,488 & 56.4756 & $>$5,500& $>$2,000\\
\bottomrule
\end{tabular}
}
\end{center}
This approach solves instances with more than 24 queens.
The use of aggressive cut generation improves the upper
bound on the objective function significantly, though
it can be observed that for values of
$n$ larger than 24 \cplex is not able to deduce the trivial upper
bound of $n$.\footnote{For the 32 queens instance the optimal solution is found
after 800 nodes, but the upper bound is still 56.1678.}
If we use the following formulation instead of constraint c2, this changes:

\medskip
%\noindent\textsf{Binary formulation B}
\lstset{language=zimpl,%
basicstyle=\sffamily\footnotesize,%
numberstyle=\sffamily\tiny\color{siennabrown},stepnumber=1}
\begin{lstlisting}[firstnumber=13]{}
subto c3: forall <i,j> in CxC do
             forall <m,n> in TABU[i,j] do x[i,j] + x[m,n] <= 1;
\end{lstlisting}

\medskip
\noindent As shown in the table below, the optimal upper bound on the objective
function is always found in the root node. This leads to a similar
situation as in the integer formulation, \ie the solution time depends
mainly on the time it needs to find the optimal solution. While
reducing the number of branch-and-bound nodes evaluated,
aggressive cut generation increases the total solution time.

With this approach instances, up to 96 queens can be solved.
At this point, the integer program gets too large to be generated.
Even though the \cplex presolve routine is able to aggregate the
constraints  again, \zimpl needs too much memory to generate the \ip.
The column labeled \emph{Pres. NZ} lists the number of non-zero entries
after the presolve procedure.

\begin{center}
{\sffamily\small
\begin{tabular}{ccrrrrrrrr}
\toprule
       &   &      &      &        & Pres. & Root &       & Time    \\
Queens & S & Vars & Cons &   NZ   &  NZ   & Node & Nodes & [s]\\
\midrule
%   8   & D &    64 &     1,456 &     2,912 &    528 &  8.0 &       0 & $<$1\\
  16   & D &   256 &    12,640 &    25,280 &  1,594 & 16.0 &       0 & $<$1\\
  32   & D & 1,024 &   105,152 &   210,304 &  6,060 & 32.0 &      58 & 5  \\
  64   & D & 4,096 &   857,472 & 1,714,944 & 23,970 & 64.0 &     110 & 60\\
  64   & C &       &           &           &        & 64.0 &      30 & 89\\
  96   & D & 9,216 & 2,912,320 & 5,824,640 & 53,829 & 96.0 &      70 & 193\\
  96   & C &       &           &           &        & 96.0 &      30 & 410\\
  96   & F &       &           &           &        & 96.0 &      69 & 66\\
\bottomrule
\end{tabular}
}
\end{center}

Finally, we will try the following set packing formulation:

\medskip
%\noindent\textsf{Set packing formulation}
\lstset{language=zimpl,%
basicstyle=\sffamily\footnotesize,%
numberstyle=\sffamily\tiny\color{siennabrown},stepnumber=1}
\begin{lstlisting}[frame=tb,firstnumber=13]{}
subto row: forall <i> in C do
   sum <i,j> in CxC : x[i,j] <= 1;

subto col: forall <j> in C do
   sum <i,j> in CxC : x[i,j] <= 1;

subto diag_row_do: forall <i> in C do
   sum <m,n> in CxC with m - i == n - 1: x[m,n] <= 1;

subto diag_row_up: forall <i> in C do
   sum <m,n> in CxC with m - i == 1 - n: x[m,n] <= 1;

subto diag_col_do: forall <j> in C do
   sum <m,n> in CxC with m - 1 == n - j: x[m,n] <= 1;

subto diag_col_up: forall <j> in C do
   sum <m,n> in CxC with card(C) - m == n - j: x[m,n] <= 1;
\end{lstlisting}

\medskip
\noindent Here again, the upper bound on the objective function is always
optimal. The size of the generated \ip is even smaller than that of
the former model after presolve. The results for different instances
size are shown in the following table:
\begin{center}
{\sffamily\small
\begin{tabular}{ccrrrrrrr}
\toprule
Queens & S & Vars & Cons    &   NZ      & Root Node & Nodes & Time [s]\\
\midrule
%   8   & D &     64 &      48 &       272 &   8.0 &     0 & $<$1\\
%  16   & D &    256 &      96 &      1056 &  16.0 &     8 & $<$1\\
%  32   & D &  1,024 &     192 &      4160 &  32.0 &     0 & $<$1\\
  64   & D &  4,096 &     384 &    16,512 &  64.0 &     0 & $<$1\\
  96   & D &  9,216 &     576 &    37,056 &  96.0 &  1680 & 331\\
  96   & C &        &         &           &  96.0 &  1200 & 338\\
  96   & F &        &         &           &  96.0 &   121 &  15\\
 128   & D & 16,384 &     768 &    65,792 & 128.0 & $>$7000 &$>$3600 &\\
 128   & F &        &         &           & 128.0 &   309 &  90\\
\bottomrule
\end{tabular}
}
\end{center}
In case of the 128 queens instance with default settings, a solution with 127 queens is found
after 90 branch-and-bound nodes, but \cplex was not able to find the
optimal solution within an hour. From the performance of the Feasible
setting, it can be presumed that generating cuts is not beneficial for
this model.



%--------------------------------------------------------------------------------------
%\section{Frequently asked questions}
%
%\input{zimplfaq}
%
%\clearpage

\section{报错信息}
以下是一份\zimpl 可能产生的无法理解的报错消息的 (希望是) 完整的列表:
% Here is a (hopefully) complete list of the incomprehensible error messages \zimpl
% can produce:

\begin{description}
%
% zimpl.c
%
\item[101 Bad filename]\ \\
  通过\code{-o}参数提供的文件名无效,可能是因为缺少文件名、
  目录名,或文件名以点开头。
  % The name given with the \code{-o} option is either missing, 
  % a directory name, or starts with a dot.
\item[102 File write error]\ \\
  写入输出文件时发生错误。错误描述将在下一行显示。有关错误含义,
  请参阅您的操作系统文档。
  % Some error occurred when writing to an output file. A description of 
  % the error follows on the next line. For the meaning 
  % consult your OS documentation.
\item[103 Output format not supported, using LP format]\ \\
  尝试选择了除\code{lp}、\code{mps}、\code{hum}、\code{rlp}或
  \code{pip}之外的格式。
  % You tried to select another format than \code{lp}, \code{mps},
  % \code{hum}, \code{rlp}, or \code{pip}.
\item[104 File open failed]\ \\
  尝试打开文件进行写入时发生错误。关于错误的描述将在下一行显示。
  有关错误信息的含义,请参阅您的操作系统文档。
  % Some error occurred when trying to open a file for writing. A description
  % of the error follows on the next line. For the meaning 
  % consult your OS documentation.
%
% inst.c
%
\item[105 Duplicate constraint name ``xxx'']\ \\
  两个\code{subto}语句具有相同的名称。
  % Two \code{subto} statements have the same name.
\item[106 Empty LHS, constraint trivially violated]\ \\
  约束的一侧为空,而另一侧不等于零。
  这种情况通常发生在对一个空的集合求和时。
  % One side of your constraint is empty and the other not equal to
  % zero. Most frequently this happens, when a set to be summed up is empty.
\item[107 Range must be $l\leq x\leq u$, or $u \geq x\geq l$]\ \\
  如果您指定范围,则必须在两侧使用相同的比较运算符。
  % If you specify a range you must have the same comparison operators
  % on both sides.
\item[108 Empty Term with nonempty LHS/RHS, constraint trivially violated]\ \\
  约束的中部为空,而约束的左侧或者右侧不等于0。
  这种情况通常发生在对一个空的集合求和时。
  % The middle of your constraint is empty and either the left- or
  % right-hand side of the range is not zero.
  % This most frequently happens, when a set to be summed up is empty.
\item[109 LHS/RHS contradiction, constraint trivially violated]\ \\
  取值范围较低的一侧大于较高的一侧,例如$15\leq x\leq 2$。
  % The lower side of your range is bigger than the upper side, e.g.
  % $15\leq x\leq 2$. 
\item[110 Division by zero]\ \\
  你在尝试除以零。这不是一个好主意。
  % You tried to divide by zero. This is not a good idea.
\item[111 Modulo by zero]\ \\
  你在尝试让一个数字参与以0为模的运算,这将是不可行的。
  % You tried to compute a number modulo zero. This does not work well.
\item[112 Exponent value \code{xxx} is too big or not an integer]\ \\
  只允许对数值计算整数次幂,且幂指数不得超过二十亿。\footnote{
    这种做法相当于是写了一个死循环\code{for(;;)},或者更确切地来说就像是在递归
    \code{void f()\{f();\}}。}
  % It is only allowed to raise a number to the power of integers. Also trying
  % to raise a number to the power of more than two billion is 
  % prohibited.\footnote{The behavior of this operation could 
  % easily be implemented as \code{for(;;)} or in a more elaborate way as 
  % \code{void f()\{f();\}}.}
\item[113 Factorial value \code{xxx} is too big or not an integer]\ \\
  你只能计算整数阶的阶乘。同时,计算一个超过二十亿的数的阶乘通常是一个坏主意。
  另请参阅报错115。
  % You can only compute the factorial of integers.
  % Also computing the factorial of a number bigger then two billion
  % is generally a bad idea. See also Error 115.
\item[114 Negative factorial value]\ \\
  你只能计算正数的阶乘。如果您需要为负数计算阶乘,请注意,
  对于所有偶数,结果将是正数,对于所有奇数,结果将是负数。
  % To compute the factorial of a number it has to be positive.
  % In case you need it for a negative number, remember that for all 
  % even numbers the outcome will be positive and for all odd number negative.
\item[115 Timeout!]\ \\
  你在尝试计算一个大于$1000!$的数。另请参阅报错112的脚注。
  % You tried to compute a number bigger than $1000!$. 
  % See also the footnote to Error 112.
\item[116 Illegal value type in min: \code{xxx} only numbers are possible]\ \\
  你在尝试计算一些字符串的最小值。
  % You tried to build the minimum of some strings.
\item[117 Illegal value type in max: \code{xxx} only numbers are possible]\ \\
  你在尝试计算一些字符串的最大值。
  % You tried to build the maximum of some strings.
\item[118 Comparison of different types]\ \\
  你在尝试比较不同类型的数据,例如比较数字和字符串。
  请注意,使用未定义的参数也可能导致此消息。
  % You tried to compare apples with oranges, i.e, numbers with
  % strings. Note that the use of an undefined parameter can also
  % lead to this message.
\item[119 \code{xxx} of sets with different dimension]\ \\
  要对两个集合应用操作\code{xxx} (取并集、差集、交集、对称差集),
  它们必须具有相同维度的元组,这也就是说,每个元组必须包含相同个数的分量。
  % To apply Operation \code{xxx} (union, minus, intersection, symmetric
  % difference) on two sets, 
  % both must have the same dimension tuples,\ie
  % the tuples must have the same number of components.
\item[120 Minus of incompatible sets]\ \\
  要对两个集合应用操作xxx (并集、差集、交集、对称差集) ,
  它们必须具有相同变量类型的元组,这也就是说,
  元组中的每个分量必须是相同的数据类型 (数字、字符串) 。
  % To apply Operation \code{xxx} (union, minus, intersection, symmetric
  % difference) on two sets, 
  % both must have tuples of the same type,\ie
  % the components of the tuples must have the same type (number,
  % string).
\item[121 Negative exponent on variable]\ \\
  变量的指数为负数。不支持这种定义方式。
  % The exponent to a variable was negative. This is not supported.
%\item[122]\ \\
\item[123 ``from'' value \code{xxx} is too big or not an integer]\ \\
  如果要生成集合,``from'' 后的数字必须是一个绝对值小于二十亿的整数。
  % To generate a set, the ``from'' number must be an integer with an
  % absolute value of less than two billion.
\item[124 ``upto'' value \code{xxx} is too big or not an integer]\ \\
  如果要生成集合,``upto'' 后的数字必须是一个绝对值小于二十亿的整数。
  % To generate a set, the ``upto'' number must be an integer with an
  % absolute value of less than two billion.
\item[125 ``step'' value \code{xxx} is too big or not an integer]\ \\
  如果要生成集合,``step'' 后的数字必须是一个绝对值小于二十亿的整数。
  % To generate a set, the ``step'' number must be an integer with an
  % absolute value of less than two billion.
\item[126 Zero ``step'' value in range]\ \\
  给定的 ``step'' (步长) 为零,因此永远无法达到 ``upto'' 的值。
  % The given ``step'' value for the generation of a set is
  % zero. So the ``upto'' value can never be reached. 
\item[127 Illegal value type in tuple: \code{xxx} only numbers are
  possible]\ \\
  \code{proj}函数中调用的选择元组 (selection tuple) 只能包含数字。
  % The selection tuple in a call to the \code{proj} function can
  % only contain numbers.
\item[128 Index value \code{xxx} in proj too big or not an integer]\ \\
  \code{proj}函数的选择元组中的值不是整数或超过了二十亿。
  % The value given in a selection tuple of a \code{proj} function is
  % not an integer or bigger than two billion.
\item[129 Illegal index \code{xxx}, set has only dimension \code{yyy}]\ \\
  选择元组中的索引值大于被索引集合中元组的维度。
  % The index value given in a selection tuple is bigger than the
  % dimension of the tuples in the set.
% Removed in 2.02
%\item[130 Duplicate index \code{xxx} for initialization]\ \\
%  In the initialization of a indexed set, two initialization elements
%  have the same index. 
%  E.g, \code{set A[] := $<$1$>$ \{ 1 \}, $<$1$>$ \{ 2 \};}
\item[131 Illegal element \code{xxx} for symbol]\footnote{\textbf{译者注}:
    报错130在版本2.02中被移除。
  }\ \\
  索引集初始化列表中使用的索引元组不是集合的索引集的成员。例如:
  \code{set A[\{ 1 to 5 \}] := $<$1$>$ \{ 1 \}, $<$6$>$ \{ 2 \};}。
  % The index tuple used in the initialization list of a index set, is
  % not member of the index set of the set.
  % E.g, \code{set A[\{ 1 to 5 \}] := $<$1$>$ \{ 1 \}, $<$6$>$ \{ 2 \};}
\item[132 Values in parameter list missing, probably wrong read
  template]\ \\
  可能是read语句的模板形如\code{"$<$1n$>$"},只有一个元组,而不是
  \code{"$<$1n$>$ 2n"}。
  % Probably the template of a read statement looks
  % like \code{"$<$1n$>$"} only having a tuple, instead of
  % \code{"$<$1n$>$ 2n"}.
\item[133 Unknown symbol \code{xxx}]\ \\
  使用了没有在作用域内定义的名称。
  % A name was used that is not defined anywhere in scope.
\item[134 Illegal element \code{xxx} for symbol]\ \\
  初始化过程中给出的索引元组不是参数的索引集的成员。
  % The index tuple given in the initialization is not member of the
  % index set of the parameter.
\item[135 Index set for parameter \code{xxx} is empty]\ \\
  尝试声明一个索引集为空的索引参数。很可能是索引集有一个\code{with}子句,
  它拒绝了所有元素。
  % The attempt was made to declare an indexed parameter with the
  % empty set as index set. Most likely the index set has a \code{with}
  % clause which has rejected all elements.
%
% space here
%
\item[139 Lower bound for integral var \code{xxx} truncated to \code{yyy}]
  (warning)\ \\
  整型变量只能有整数边界,给定非整数边界则被调整为整数边界。
  % An integral variable can only have an integral bound. So the given
  % non integral bound was adjusted.
\item[140 Upper bound for integral var \code{xxx} truncated to \code{yyy}]
  (warning)\ \\
  整型变量只能有整数边界,给定非整数边界则被调整为整数边界。
  % An integral variable can only have an integral bound. So the given
  % non integral bound was adjusted.
\item[141 Infeasible due to conflicting bounds for var \code{xxx}]\ \\
  变量的上界小于下界。
  % The upper bound given for a variable was smaller than the lower bound.
\item[142 Unknown index \code{xxx} for symbol \code{yyy}]\ \\
  符号的索引集不包含给定的索引元组。
  % The index tuple given is not member of the index set of the symbol.
\item[143 Size for subsets \code{xxx} is too big or not an integer]\ \\
  给定的用于生成的子集的基数必须是一个小于二十亿的整数。
  % The cardinality for the subsets to generate must be given as an
  % integer smaller than two billion.
\item[144 Tried to build subsets of empty set]\ \\
  用于生成子集的集合是空集。
  % The set given to build the subsets of, was the empty set.
\item[145 Illegal size for subsets \code{xxx}, should be between 1 
  and \code{yyy}]\ \\
  用于生成的子集的基数必须在1和基集合的基数 (集合元素的个数) 之间。
  % The cardinality for the subsets to generate must be between 1 
  % and the cardinality of the base set.
\item[146 Tried to build powerset of empty set ]\ \\
  用于生成幂集的集合是空集。
  % The set given to build the powerset of, was the empty set.
%
% iread.c
%
\item[147 use value \code{xxx} is too big or not an integer]\ \\
  use 参数的值必须是一个小于二十亿的整数。
  % The use value must be given as an integer smaller than two billion.
\item[148 use value \code{xxx} is not positive]\ \\
  use 参数不能为负数或 0。
  % Negative or zero values for the use parameter are not allowed.
\item[149 skip value \code{xxx} is too big or not an integer]\ \\
  skip 参数的值必须是一个小于二十亿的整数。
  % The skip value must be given as an integer smaller than two billion.
\item[150 skip value \code{xxx} is not positive]\ \\
  skip 参数不能为负数或 0。
  % Negative or zero values for the skip parameter are not allowed.
\item[151 Not a valid read template]\ \\
  read 模板必须是类似\code{"$<$1n,2n$>$"}的格式,按顺序必须有 $<$ 和 $>$ 。
  % A read template must look something like \code{"$<$1n,2n$>$"}.
  % There have to be a $<$ and a $>$ in this order.
\item[152 Invalid read template syntax]\ \\
  除了任何分隔符 (如\code{$<$}、\code{$>$}和逗号) 之外,
  模板必须由数字字符对组成,如\code{1n}, \code{3s}。
  % Apart from any delimiters like \code{$<$}, \code{$>$}, and commas a
  % template must consists of number character pairs like \code{1n}, \code{3s}.
\item[153 Invalid field number \code{xxx}]\ \\
  模板中的字段编号必须在1到255之间。
  % The field numbers in a template have to be between 1 and 255.
\item[154 Invalid field type \code{xxx}]\ \\
  字段类型只能是\code{n}和\code{s}。
  % The only possible field types are \code{n} and \code{s}.
\item[155 Invalid read template, not enough fields]\ \\
  分隔符之间至少要有一个字段。
  % There has to be at least one field inside the delimiters.
\item[156 Not enough fields in data]\ \\
  模板指定的字段编号高于数据中实际找到的字段数。
  % The template specified a field number that is higher than the actual
  % number of field found in the data. 
\item[157 Not enough fields in data (value)]\ \\
  模板指定的字段编号高于数据中实际找到的字段数。
  错误发生在值字段中的索引元组之后。
  % The template specified a field number that is higher than the actual
  % number of field found in the data. The error occurred after the 
  % index tuple in the value field.
%\item[158 ]\ \\

%
% code.c
%  
\item[159 Type error, expected \code{xxx} got \code{yyy}]\ \\
  找到的数据类型不是预期的类型,例如从数字中减去字符串将导致此消息。
  % The type found was not the expected one, e.g. subtracting 
  % a string from a number would result in this message.
%
% elem.c
% 
\item[160 Comparison of elements with different types \code{xxx} /
  \code{yyy}]\ \\
  比较两个不同元组中的元素时,发现它们的变量类型不同。
  % Two elements from different tuples were compared and found to be 
  % of different types. 
%
% load.c
%
\item[161 Line \code{xxx}: Unterminated string]\ \\
  该行具有奇数个\code{"}字符。某一字符串已开始,但未封闭。
  % This line has an odd number of \code{"} characters. 
  % A String was started, but not ended.
\item[162 Line \code{xxx}: Trailing \code{"yyy"} ignored] (warning)\ \\
  文件中的最后一个分号后发现有内容。
  % Something was found after the last semicolon in the file.
\item[163 Line \code{xxx}: Syntax Error]\ \\
  某一语句未以如下关键字开头:
  \code{set}, \code{param}, \code{var}, \code{minimize}, 
  \code{maximize}, \code{subto}, or \code{do}
  % A new statement was not started with one of the keywords:
  % \code{set}, \code{param}, \code{var}, \code{minimize}, 
  % \code{maximize}, \code{subto}, or \code{do}.
%
% set.c
%
\item[164 Duplicate element \code{xxx} for set rejected] (warning)\ \\
  尝试向集合中添加一个已存在的元素。
  % An element was added to a set that was already in it.
\item[165 Comparison of different dimension sets] (warning)\ \\
  比较两个集合,但它们的维度元组不同。
  (这意味着除了空集之外,它们永远也不可能相等。)
  % Two sets were compared, but have different dimension tuples.
  % (This means they never had a chance to
  % be equal, other than being empty sets.)
%
% symbol.c
%
\item[166 Duplicate element \code{xxx} for symbol \code{yyy} rejected]
  (warning)\ \\
  尝试向\code{yyy}中添加一个已存在的元素。
  % An element that was already there was added to a symbol.
%
% tuple.c
%
\item[167 Comparison of different dimension tuples] (warning)\ \\
  尝试比较两个维度不同的元组。
  % Two tuples with different dimensions were compared.
%
% zimpl.c
%
\item[168 No program statements to execute]\ \\
  在加载的文件中没有找到\zimpl 语句。
  % No \zimpl statements were found in the files loaded.
%
% code.c
%
\item[169 Execute must return void element]\ \\
  这不应该发生。如果您遇到此错误,请将.zpl文件发送至\url{mailto:koch@zib.de}。
  % This should not happen. If you encounter
  % this error please email the \code{.zpl} file to \url{mailto:koch@zib.de}.
%
% inst.c
%
\item[170 Uninitialized local parameter \code{xxx} in call of
  define \code{yyy}]\ \\
  调用定义 (define) 时,其中一个参数是一个``名称''
  (变量的名称),但没有为其定义值。
  % A define was called and one of the arguments was a ``name'' 
  % (of a variable) for which no value was defined.
\item[171 Wrong number of arguments (\code{xxx} instead of \code{yyy})
  for call of define \code{zzz}]\ \\
  调用定义 (define) 时,参数数量与定义的不同。
  % A define was called with a different number of arguments than in
  % its definition.
\item[172 Wrong number of entries (\code{xxx}) in table line, 
  expected \code{yyy} entries]\ \\
  在参数的初始化表中,每一行输入的条目 (entries) 的个数
  必须与表的索引行 (第一行) 相同。
  % Each line of a parameter initialization table must have
  % exactly the same number of entries as the index (first) line of
  % the table.
\item[173 Illegal type in element \code{xxx} for symbol]\ \\
  一个参数表只能有一个数据类型,要么是数字,要么是字符串。
  在初始化过程中同时指定了两种类型。
  % A parameter can only have a single value type. Either numbers or
  % strings. In the initialization both types were present.
%
% iread.c
%
\item[174 Numeric field \code{xxx} read as \code{"yyy"}. This is not a
  number]\ \\
  试图读取一个在模板中指定为 `n' 的字段,但读取的内容不是有效的数字。
  % It was tried to read a field with an 'n' designation in the template,
  % but what was read is not a valid number.
%
% zimpl.c
%
\item[175 Illegal syntax for command line define \code{"xxx"} --
  ignored] (warning)\\
  使用命令行选项 \code{-D} 指定的参数必须匹配 \code{name=value} 的格式。
  其中 \code{name} 必须是一个合法的标识符,这也就是说,它必须以字母开头,
  并且只能由字母、数字和下划线组成。
  % A parameter definition using the command line \code{-D} flag, must
  % have the form \code{name=value}. The \code{name} must be a legal
  % identifier, \ie it has to start with a letter and may consist only out
  % of letters and numbers including the underscore. 
%
% vinst.c
%
\item[176 Empty LHS, in Boolean constraint] (warning)\ \\
  (逻辑约束的) 左半边为空,这也就是说,包含变量的项式为空。
  % The left hand side, \ie the term with the variables, is empty. 
\item[177 Boolean constraint not all integer]\ \\
  在逻辑约束中,不允许出现连续型 (实型) 变量。 
  % No continuous (real) variables are allowed in a Boolean constraint.
\item[178 Conditional always true or false due to bounds] (warning)\ \\
  由于变量的上下界,逻辑约束的全部或部分始终为真或者假。
  % All or part of a Boolean constraint are always either true or
  % false, due to the bounds of variables.
\item[179 Conditional only possible on bounded constraints]\ \\
  在某个逻辑约束中至少存在一个变量没有有限边界。
  % A Boolean constraint has at least one variable without finite bounds.
\item[180 Conditional constraint always true due to bounds] (warning)\ \\
  条件约束的结果部分恒为真。这是由于包含了变量的上下界。
  % The result part of a conditional constraint is always true anyway. 
  % This is due to the bounds of the variables involved.
\item[181 Empty LHS, not allowed in conditional constraint]\ \\
  条件约束的结果部分不能为空。
  % The result part of a conditional constraint may not be empty.
\item[182 Empty LHS, in variable vabs]\ \\
  \code{vabs} 函数的参数当中没有变量。要么所有的值都会被归零,
  要么只需使用 \code{abs} 即可。
  % There are no variables in the argument to a \code{vabs} function.
  % Either everything is zero, or just use \code{abs}.
\item[183 vabs term not all integer]\ \\
  \code{vabs} 函数的声明中包含了非整型变量。由于数值的原因 (numerical reason),
  不允许将连续型变量作为参数传递给 \code{vabs}。
  % There are non integer variables in the argument to a \code{vabs} function.
  % Due to numerical reasons continuous variables are not allowed as
  % arguments to \code{vabs}. 
\item[184 vabs term not bounded]\ \\
  \code{vabs} 函数内的项式中至少包含一个无界变量。
   The term inside a \code{vabs} has at least one unbounded variable.
\item[185 Term in Boolean constraint not bounded]\ \\
  \code{vif} 内的项式中至少包含一个无界变量。
  % The term inside a \code{vif} has at least one unbounded variable.
%
% inst.c
%
\item[186 Minimizing over empty set -- zero assumed] (warning)\ \\
  取极小值时的索引表达式为空,因为该表达式的值为零。\footnote{
    \textbf{译者注}:我确实不知道这句话是什么意思,所以此处是照着字面翻译的。}
  % The index expression for the minimization was empty. The result
  % used for this expression was zero.
\item[187 Maximizing over empty set -- zero assumed] (warning)\ \\
  取极大值时的索引表达式为空,因为该表达式的值为零。
  % The index expression for the maximization was empty. The result
  % used for this expression was zero.
\item[188 Index tuple has wrong dimension]\ \\
  索引元组中的元素数量与被索引集合的元组索引的维度不一致。导致此错误的原因,
  既可能是因为在参数初始化列表中,条目的索引元组的维度与该参数的索引集合维度不同;
  也可能是因为在集合初始化列表中,条目的索引元组维度与该集合的索引集的维度不同。
  如果您使用 \code{powerset} 或 \code{subset} 指令,则索引集必须是一维的。
  % The number of elements in an index tuple is different from the
  % dimension of the tuples in the set that is indexed.
  % This might occur, when the index tuple of an entry in a parameter initialization list has
  % not the same dimension as the indexing set of the parameter. 
  % This might occur, when the index tuple of an entry in a set
  % initialization list has
  % not the same dimension as the indexing set of the set. 
  % If you use a \code{powerset} or \code{subset} instruction, the index
  % set has to be one dimension.
\item[189 Tuple number \code{xxx} is too big or not an integer]\ \\
  给定的元组的编号必须是小于二十亿的整数。
  % The tuple number must be given as an integer smaller than two
  % billion.
\item[190 Component number \code{xxx} is too big or not an integer]\ \\
  给定的分量的编号必须是小于二十亿的整数。
  % The component number must be given as an integer smaller than two
  % billion.
\item[191 Tuple number \code{xxx} is not a valid value between 1..\code{yyy}]\ \\
  元组的编号必须在 1 和集合的基数之间。
  % The tuple number must be between one and the cardinality of the set.
\item[192 Component number \code{xxx} is not a valid value between 1..\code{yyy}]\ \\
  分量的编号必须在 1 和集合的维数之间。
  % The component number must be between one and the dimension of the set.
\item[193 Different dimension tuples in set initialization]\ \\
  同一列表中的元组的维度不同。
  % The tuples that should be part of the list have different dimension.
\item[195 Genuine empty set as index set] (warning)\ \\
  某一索引集索引的被索引集合恒为空集。
  % The set of an index set is always the empty set.
\item[197 Empty index set for set]\ \\
  某一集合的索引集为空集。
  % The index set for a set is empty.
\item[198 Incompatible index tuple]\ \\
  给定的索引元组具有固定的分量,
  该分量的数据类型与集合中的元组的同一分量的数据类型不一致。
  % The index tuple given had fixed components. The type of such a
  % component was not the same as the type of the same component of tuples
  % from the set.
\item[199 Constants are not allowed in SOS declarations]\ \\
  在声明 SOS 时,权重只能与变量一同使用。单独使用的权重没有意义。
  % When declaring an SOS, weights are only allowed together with
  % variables. A weight alone does not make sense.
\item[200 Weights are not unique for SOS \code{xxx} (warning)]\ \\
  特殊有序集中分配给变量的所有权重都必须是唯一的。
  % All weights assigned to variables in an special ordered set have to
  % be unique. 
\item[201 Invalid read template, only one field allowed]\ \\
  读取一个单一的参数值时,read 模板必须只包含一个单一的字段规则。
  % When reading a single parameter value, the read template must
  % consist of a single field specification.
\item[202 Indexing over empty set] (warning)\ \\
  索引集是空的。
  % The indexing set turns out to be empty.
\item[203 Indexing tuple is fixed] (warning)\ \\
  某个索引表达式产生的索引集是一个固定的集合。
  其结果是只有这一个 (对应的) 元素会被搜索到。
  % The indexing tuple of an index expression is completely fixed. As a
  % result only this one element will be searched for.
\item[204 Random function parameter minimum= \code{xxx} $>=$ maximum=
  \code{yyy}]\ \\
  \code{random} 函数的第二个参数必须严格大于第一个参数。
  % The second parameter to the function \code{random} has to be
  % strictly greater than the first parameter.
\item[205 \code{xxx} excess entries for symbol \code{yyy} ignored ]
  (warning)\ \\
  读取符号 \code{yyy} 中的数据时,文件中存在的条目数比符号的索引数多 
  \code{xxx} 个。多余的条目会被忽略。
  % When reading the data for symbol \code{yyy} there were 
  % \code{xxx} more entries in the file than indices for the symbol.
  % The excess entries were ignored.  
\item[206 argmin/argmax over empty set] (warning)\ \\
  \code{argmin} 或 \code{argmax} 的索引表达式为空。返回的结果为空集。
  % The index expression for the \code{argmin} or \code{argmax} was
  % empty. The result is the empty set.
\item[207 ``size'' value \code{xxx} is too big or not an integer]\ \\
   The size argument for an \code{argmin} or \code{argmax} function
   must be an integer with an absolute value of less than two billion.
\item[208 ``size'' value \code{xxx} not >= 1]\ \\
   The size argument for an \code{argmin} or \code{argmax} function
   must be at least one, since it represents the maximum cardinality
   of the resulting set.
\item[209 MIN of set with more than one dimension]\ \\
   The expressions \code{min(A)} is only allowed if the elements of 
   set A consist of 1-tuples containing numbers.  
\item[210 MAX of set with more than one dimension]\ \\
   The expressions \code{max(A)} is only allowed if the elements of 
   set A consist of 1-tuples containing numbers.  
\item[211 MIN of set containing non number elements]\ \\
   The expressions \code{min(A)} is only allowed if the elements of 
   set A consist of 1-tuples containing numbers.  
\item[212 MAX of set containing non number elements]\ \\
   The expressions \code{max(A)} is only allowed if the elements of 
   set A consist of 1-tuples containing numbers.  
\item[213 More than 65535 input fields in line \code{xxx} of
   \code{yyy} (warning)]\ \\
   Input data beyond field number 65535 in line \code{xxx} of file
   \code{yyy} are ignored. Insert some newlines into your data!
\item[214 Wrong type of set elements -- wrong read template?]\ \\
   Most likely you have tried read in a set from a stream using
   \code{"n+"} instead of \code{"<n+>"} in the template. 
\item[215 Startvals violate constraint, \ldots (warning)]\ \\
   If the given startvals are summed up, they violate the
   constraint. Details about the sum of the LHS and the RHS are given
   in the message.
\item[216 Redefinition of parameter \code{xxx} ignored]\ \\
   A parameter was declared a second time with the same name. The
   typical use would be to declare default values for a parameter in
   the \zimpl file and override them by command-line defined.
\item[217 begin value \code{xxx} in substr too big or not an integer]\ \\
   The begin argument for an \code{substr} function
   must be an integer with an absolute value of less than two billion.
\item[218 length value \code{xxx} in substr too big or not an integer]\ \\
   The length argument for an \code{substr} function
   must be an integer with an absolute value of less than two billion.
\item[219 length value \code{xxx} in substr is negative]\ \\
   The length argument for an \code{substr} function
   must be greater or equal to zero.
\item[220 Illegal size for subsets \code{xxx}, should be between \code{yyy} 
  and \code{zzz}]\ \\
  The cardinality of the subsets to generate must be between the
  given lower bound and the cardinality of the base set.
%\item[221 The objective function has to be linear or quadratic]\ \\
%   Only objective functions with linear or quadratic constraints are allowed.
% vinst.c
\item[222 Term inside a then or else constraint not linear]\ \\
  The term inside a \code{then} or \code{else} is not linear.
% inst.c
\item[223 Objective function \code{xxx} overwrites existing one
  (warning)]\ \\
  Another objective function declaration has been encountered,
  superceeding the previous one. 
% xlpglue.c
\item[301 variable priority has to be integral] (warning)\ \\
   If branching priorities for variables are given, these have to be integral.
\item[302 SOS priority has to be integral] (warning)\ \\
   If SOS priorities are given, these have to be integral.
   %
\item [401 Slack too large (\code{xxx}) for QUBO conversion]
   The upper bound of the slack needed for this constraint is more then
   $2^{30}$.   
\item [403 Non linear term can't be converted to QUBO]\ \\
  Only linear terms can be automatically converted into a QUBO.
\item [404 Non linear expressions can't be converted to QUBO]\ \\
  Only linear expressions without indicator constraints can be
  automatically converted into a QUBO. 
\item[600 File format can only handle linear and quadratic constraints
         (warning)]\ \\
  The chosen file format can currently only handle linear and
  quadratic constraints. Higher degree constraints were ignored.
\item[601 File format can only handle binary variables] (warning)]\ \\
  The chosen file format can only handle binary variables. 
  Variables of other types will be ignored.
\item[602 QUBO file format can only handle linear and quadratic term
         (warning)]\ \\
  By definition the QUBO file format can only handle linear and
  quadratic terms. No file can be written.
%
% space here
%
% 
% numbgmp.c
%
\item[700 log(): \code{OS specific domain or range error message}]\ \\
  Function \code{log} was called with a zero or negative argument, or
the argument was too small to be represented as a \code{double}.
\item[701 sqrt(): \code{OS specific domain error message}]\ \\
  Function \code{sqrt} was called with a negative argument.
\item[702 ln(): \code{OS specific domain or range error message}]\ \\
  Function \code{ln} was called with a zero or negative argument, or
  the argument was too small to be represented as a \code{double}.
%\item[]\ \\
%\item[]\ \\
%\item[]\ \\
%\item[]\ \\
%\item[]\ \\
%\item[]\ \\
%\item[]\ \\
% mmlscan.l
\item[800 parse error: expecting \code{xxx} (or \code{yyy})]\ \\
  Parsing error. What was found was not what was expected.
  The statement you entered is not valid.
\item[801 Parser failed]\ \\
  The parsing routine failed. This should not happen. If you encounter
  this error, please email the \code{.zpl} file to \url{mailto:koch@zib.de}.
\item[802 Regular expression error]\ \\
  A regular expression given to the \code{match} parameter of a
  \code{read} statement was not valid. See error messages for details.
\item[803 String too long \code{xxx} $>$ \code{yyy}]\ \\
  The program encountered a string which is larger than 1 GB. 
%
% inst.c
%
\item[900 Check failed!]\ \\
  A \code{check} instruction did not evaluate to true. 
\end{description}



\nocite{*}
\small
\bibliographystyle{alpha}
\bibliography{zimpl}
\end{document}
